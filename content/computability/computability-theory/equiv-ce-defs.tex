% Part: computability
% Chapter: computability-theory
% Section: equiv-ce-defs

\documentclass[../../../include/open-logic-section]{subfiles}

\begin{document}

\olsection[Equivalent Definitions of C. E. Sets]{cmp}{thy}{eqc}
  {Equivalent Defininitions of Computably Enumerable Sets}


The following gives a number of important equivalent statements of
what it means to be computably enumerable.

\begin{thm}
\ollabel{thm:ce-equiv}
Let $S$ be a set of natural numbers. Then the following are
equivalent:
\begin{enumerate}
\item $S$ is computably enumerable.
\item $S$ is the range of a \emph{partial} computable function.
\item $S$ is empty or the range of a primitive recursive function.
\item $S$ is the \emph{domain} of a partial computable function.
\end{enumerate}
\end{thm}

\begin{explain}
The first three clauses say that we can equivalently take any non-empty
computably enumerable set to be enumerated by either a computable
function, a partial computable function, or a primitive recursive
function. The fourth clause tells us that if $S$ is computably
enumerable, then for some index $e$,
\[
S = \Setabs{x}{\cfind{e}(x) \fdefined}.
\]
In other words, $S$ is the set of inputs on for which the computation
of $\cfind{e}$ halts. For that reason, computably enumerable sets are
sometimes called \emph{semi-decidable}: if a number is in the set, you
eventually get a ``yes,'' but if it isn't, you never get a ``no''!{}
\end{explain}

\begin{proof}
Since every primitive recursive function is computable and every
computable function is partial computable, (3) implies (1) and (1)
implies~(2). (Note that if $S$ is empty, $S$ is the range of the
partial computable function that is nowhere defined.) If we show that
(2) implies (3), we will have shown the first three clauses
equivalent.

So, suppose $S$ is the range of the partial computable function
$\cfind{e}$. If $S$ is empty, we are done. Otherwise, let $a$ be any
element of~$S$. By Kleene's normal form theorem, we can write
\[
\cfind{e}(x) = U (\umin{s}{T(e, x, s)}).
\]
In particular, $\cfind{e}(x) \fdefined$ and $= y$ if and only if there
is an $s$ such that $T(e, x, s)$ and $U(s) = y$. Define $f(z)$ by
\[
f(z) = \begin{cases}
U((z)_1) & \text{if $T(e, (z)_0, (z)_1)$} \\
a        & \text{otherwise.}
\end{cases}
\]
Then $f$ is primitive recursive, because $T$ and $U$
are. \iftag{TMs}{Expressed in terms of Turing machines, if $z$ codes a
  pair $\tuple{(z)_0, (z)_1}$ such that $(z)_1$ is a halting
  computation of machine $e$ on input $(z)_0$, then $f$ returns the
  output of the computation; otherwise, it returns $a$.}

We need to show that $S$ is the range of $f$, i.e., for any natural
number $y$, $y \in S$ if and only if it is in the range of~$f$. In the
forwards direction, suppose $y \in S$. Then $y$ is in the range of
$\cfind{e}$, so for some $x$ and $s$, $T(e,x,s)$ and $U(s) = y$; but then
$y = f(\tuple{x,s})$. Conversely, suppose $y$ is in the range of
$f$. Then either $y = a$, or for some $z$, $T(e,(z)_0,(z)_1)$ and
$U((z)_1) = y$. Since, in the latter case, $\cfind{e}(x) \fdefined = y$,
either way, $y$ is in $S$.

(The notation $\cfind{e}(x) \fdefined = y$ means ``$\cfind{e}(x)$ is
defined and equal to $y$.'' We could just as well use $\cfind{e}(x) =
y$, but the extra arrow is sometimes helpful in reminding us that we
are dealing with a partial function.)

To finish up the proof of \olref{thm:ce-equiv}, it suffices to show
that (1) and (4) are equivalent. First, let us show that (1) implies
(4). Suppose $S$ is the range of a computable function~$f$, i.e.,
\[
S = \Setabs{y}{\text{for some $x$,} f(x) = y}.
\]
Let
\[
g(y) = \umin{x}{f(x) = y}.
\]
Then $g$ is a partial computable function, and $g(y)$ is defined if
and only if for some $x$, $f(x) = y$. In other words, the domain of
$g$~is the range of~$f$. \iftag{TMs}{Expressed in terms of Turing
  machines: given a Turing machine $F$ that enumerates the elements of
  $S$, let $G$ be the Turing machine that semi-decides $S$ by
  searching through the outputs of~$F$ to see if a given element is in
  the set.}{}

Finally, to show (4) implies (1), suppose that $S$~is the domain of the
partial computable function~$\cfind{e}$, i.e.,
\[
S = \Setabs{x}{\cfind{e}(x) \fdefined}.
\]
If $S$ is empty, we are done; otherwise, let $a$ be any element
of~$S$. Define $f$ by
\[
f(z) = \begin{cases}
(z)_0 & \text{if $T(e,(z)_0,(z)_1)$} \\
a & \text{otherwise.}
\end{cases}
\]
Then, as above, a number $x$ is in the range of~$f$ if and only if
$\cfind{e}(x) \fdefined$, i.e., if and only if $x \in S$. \iftag{TMs}{Expressed
in terms of Turing machines: given a machine $M_e$ that semi-decides
$S$, enumerate the elements of $S$ by running through all possible
Turing machine computations, and returning the inputs that correspond
to halting computations.}{}
\end{proof}

The fourth clause of \olref{thm:ce-equiv} provides us with a
convenient way of enumerating the computably enumerable sets: for each
$e$, let $W_e$ denote the domain of $\cfind{e}$. Then if $A$ is any
computably enumerable set, $A = W_e$, for some $e$.

The following provides yet another characterization of the computably
enumerable sets.

\begin{thm}
\ollabel{thm:exists-char}
A set $S$ is computably enumerable if and only if there is a
computable relation $\Atom{R}{x,y}$ such that
\[
S = \Setabs{ x }{ \lexists[y][\Atom{R}{x,y}] }.
\]
\end{thm}

\begin{proof}
In the forward direction, suppose $S$ is computably
enumerable. Then for some $e$, $S = W_e$. For this value of $e$
we can write $S$ as
\[
S = \Setabs{ x }{ \lexists[y][T(e, x, y)] }.
\]
In the reverse direction, suppose $S = \Setabs{ x }{
  \lexists[y][\Atom{R}{x, y}] }$. Define $f$~by
\[
f(x) \simeq \umin{y}{Atom{R}{x,y}}.
\]
Then $f$ is partial computable, and $S$ is the domain of~$f$.
\end{proof}


\end{document}
