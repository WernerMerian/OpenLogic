% Part: computability
% Chapter: computability-theory
% Section: halting-problem

\documentclass[../../../include/open-logic-section]{subfiles}

\begin{document}

\olsection{cmp}{thy}{hlt}{The Halting Problem}

Since, in our construction, $\fn{Un}(k,x)$ is defined if and only if
the computation of the function coded by~$k$ produces a value for
input~$x$, it is natural to ask if we can decide whether this is the
case. And in fact, it is not. For the Turing machine model of
computation, this means that whether a given Turing
machine halts on a given input is computationally undecidable. The
following theorem is therefore known as the ``undecidability of the
halting problem.'' We will provide two proofs below. The first
continues the thread of our previous discussion, while the second is
more direct.

\begin{thm}
\ollabel{thm:halting-problem}
Let
\[
h(k, x)  =
\begin{cases}
1 & \text{if $\fn{Un}(k,x)$ is defined} \\
0 & \text{otherwise.}
\end{cases}
\]
Then $h$ is not computable.
\end{thm}

\begin{proof}
If $h$ were computable, we would have a universal computable function,
as follows. Suppose $h$ is computable, and define
\[
\fn{Un'}(k,x) =
\begin{cases}
fn{Un}(k,x) & \text{if $h(k,x) = 1$} \\
0 & \text{otherwise.}
\end{cases}
\]
But now $\fn{Un'}(k, x)$ is a total function, and is computable if~$h$
is.  For instance, we could define $g$ using primitive recursion, by
\begin{align*}
g(0, k, x) & \simeq 0 \\
g(y+1, k, x) & \simeq \fn{Un}(k,x);
\end{align*}
then
\[
\fn{Un'}(k,x) \simeq g(h(k,x),k,x).
\]
And since $\fn{Un'}(k,x)$ agrees with $\fn{Un}(k,x)$ wherever the
latter is defined, $\fn{Un'}$ is universal for those partial
computable functions that happen to be total.  But this contradicts
\olref[nou]{thm:no-univ}.
\end{proof}

\begin{proof}
Suppose $h(k,x)$ were computable. Define the function $g$ by
\[
g(x) =
\begin{cases}
  0                & \text{if $h(x,x) = 0$} \\
  \text{undefined} & \text{otherwise.}
\end{cases}
\]
The function $g$ is partial computable; for example, one can define it
as $\umin{y}{h(x,x) = 0}$. So, for some $k$, $g(x) \simeq \fn{Un}(k, x)$ for
every~$x$. Is $g$ defined at $k$?  If it is, then, by the definition
of $g$, $h(k,k) = 0$. By the definition of $f$, this means that
$\fn{Un}(k, k)$ is undefined; but by our assumption that $g(k) \simeq
\fn{Un}(k, x)$ for every $x$, this means that $g(k)$ is undefined, a
contradiction. On the other hand, if $g(k)$ is undefined, then $h(k,k)
\neq 0$, and so $h(k,k) = 1$. But this means that $\fn{Un}(k, k)$ is
defined, i.e., that $g(k)$ is defined.
\end{proof}

\begin{tagblock}{TMs}
\begin{explain}
We can describe this argument in terms of Turing machines.  Suppose
there were a Turing machine $H$ that took as input a description
of a Turing machine $K$ and an input $x$, and decided whether or not
$K$ halts on input $x$. Then we could build another Turing machine $G$
which takes a single input $x$, calls $H$ to decide if machine $x$
halts on input $x$, and does the opposite. In other words, if $H$
reports that $x$ halts on input $x$, $G$ goes into an infinite loop,
and if $H$ reports that $x$ doesn't halt on input $x$, then $G$ just
halts. Does $G$ halt on input $G$? The argument above shows that it
does if and only if it doesn't---a contradiction. So our supposition that
there is a such Turing machine~$H$, is false.
\end{explain}
\end{tagblock}

\end{document}
