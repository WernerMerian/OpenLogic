% Part: first-order-logic
% Chapter: beyond
% Section: other-logics

\documentclass[../../../include/open-logic-section]{subfiles}

\begin{document}

\olsection{fol}{byd}{oth}{Other Logics}

As you may have gathered by now, it is not hard to design a new logic.
You too can create your own a syntax, make up a deductive system, and
fashion a semantics to go with it. You might have to be a bit clever
if you want the !!{derivation} system to be complete for the semantics, and it
might take some effort to convince the world at large that your logic
is truly interesting. But, in return, you can enjoy hours of good,
clean fun, exploring your logic's mathematical and computational
properties.

Recent decades have witnessed a veritable explosion of formal logics.
Fuzzy logic is designed to model reasoning about vague
properties. Probabilistic logic is designed to model
reasoning about uncertainty. Default logics and nonmonotonic logics
are designed to model defeasible forms of reasoning, which is
to say, ``reasonable'' inferences that can later be overturned in the
face of new information. There are epistemic logics, designed to
model reasoning about knowledge; causal logics, designed to
model reasoning about causal relationships; and even
``deontic'' logics, which are designed to model reasoning
about moral and ethical obligations. Depending on whether the primary
motivation for introducing these systems is philosophical,
mathematical, or computational, you may find such creatures studies
under the rubric of mathematical logic, philosophical logic,
artificial intelligence, cognitive science, or elsewhere.

The list goes on and on, and the possibilities seem endless. We
may never attain Leibniz' dream of reducing all of human reason to
calculation---but that can't stop us from trying.

\end{document}
