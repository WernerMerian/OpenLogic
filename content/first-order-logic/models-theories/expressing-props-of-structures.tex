% Part: first-order-logic
% Chapter: models-theories
% Section: expressing-props-of-structures

\documentclass[../../../include/open-logic-section]{subfiles}

\begin{document}

\olsection{fol}{mat}{exs}{Expressing Properties of \printtoken{P}{structure}}

\begin{explain}
It is often useful and important to express conditions on
functions and relations, or more generally, that the functions and
relations in a structure satisfy these conditions.  For instance, we
would like to have ways of distinguishing those !!{structure}s for a
language which ``capture'' what we want the !!{predicate}s to ``mean''
from those that do not.  Of course we're completely free to specify
which !!{structure}s we ``intend,'' e.g., we can specify that the
interpretation of the !!{predicate}~$\le$ must be an ordering, or that
we are only interested in interpretations of~$\Lang L$ in which the
domain consists of sets and $\Obj \in$ is interpreted by the ``is
!!a{element} of'' relation.  But can we do this with !!{sentence}s of
the language?  In other words, which conditions on
!!a{structure}~$\Struct M$ can we express by !!a{sentence} (or perhaps
a set of !!{sentence}s) in the language of~$\Struct M$?  There are
some conditions that we will not be able to express.  For instance,
there is no sentence of~$\Lang L_A$ which is only true in a
!!{structure}~$\Struct M$ if $\Domain M = \Nat$.  We cannot express
``the domain contains only natural numbers.''  But there are
``structural properties'' of !!{structure}s that we perhaps can
express.  Which properties of !!{structure}s can we express by
!!{sentence}s?  Or, to put it another way, which collections of
!!{structure}s can we describe as those making !!a{sentence} (or set
of !!{sentence}s) true?
\end{explain}

\begin{defn}[Model of a set]
Let $\Gamma$ be a set of !!{sentence}s in a language~$\Lang L$.  We
say that !!a{structure}~$\Struct M$ \emph{is a model of}~$\Gamma$ if
$\Sat{M}{!A}$ for all $!A \in \Gamma$.
\end{defn}

\begin{ex}
The sentence $\lforall[x][x \le x]$ is true in~$\Struct M$ iff
$\Assign{\le}{M}$ is a reflexive relation.  The sentence
$\lforall[x][\lforall[y][((x \le y \land y \le x) \lif x = y)]]$ is
true in~$\Struct M$ iff $\Assign{\le}{M}$ is anti-symmetric.  The
sentence $\lforall[x][\lforall[y][\lforall[z][((x \le y \land y \le z)
      \lif x \le z)]]]$ is true in~$\Struct M$ iff $\Assign{\le}{M}$
is transitive.  Thus, the models of
\begin{align*}
\{\quad &\lforall[x][x \le x], \\
   & \lforall[x][\lforall[y][((x \le y \land y \le
    x) \lif x = y)]], \\
   &\lforall[x][\lforall[y][\lforall[z][((x \le y
      \land y \le z) \lif x \le z)]]] \quad \}
\end{align*}
are exactly those structures in which~$\Assign{\le}{M}$ is reflexive,
anti-symmetric, and transitive, i.e., a partial order.  Hence, we can take
them as axioms for the \emph{first-order theory of partial orders}.
\end{ex}

\end{document}
