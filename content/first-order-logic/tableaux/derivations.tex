% Part: first-order-logic
% Chapter: tableaux
% Section: derivations

\documentclass[../../../include/open-logic-section]{subfiles}

\begin{document}

\iftag{FOL}
      {\olsection{fol}{tab}{der}{\usetoken{P}{tableau}}}
      {\olsection{pl}{tab}{der}{\usetoken{P}{tableau}}}

\begin{explain}
We've said what an assumption is, and we've given the rules of
inference.  !!^{tableau}s are inductively generated from these: each
!!{tableau} either is a single branch consisting of one or more
assumptions, or it results from !!a{tableau} by applying one of the
rules of inference on a branch.
\end{explain}

\begin{defn}[!!^{tableau}]
!!^a{tableau} for assumptions $\sFmla{S_1}{!A_1}$, \dots,
$\sFmla{S_n}{!A_n}$ (where each $S_i$ is either $\True$ or~$\False$) is
a finite tree of !!{signed formula}s satisfying the following conditions:
\begin{enumerate}
\item The $n$ topmost !!{signed formula}s of the tree are
  $\sFmla{S_i}{!A_i}$, one below the other.
\item Every !!{signed formula} in the tree that is not one of the
  assumptions results from a correct application of an inference rule
  to !!a{signed formula} in the branch above it.
\end{enumerate}
A branch of !!a{tableau} is \emph{closed} iff it contains both
$\sFmla{\True}{!A}$ and~$\sFmla{\False}{!A}$, and \emph{open}
otherwise. !!^a{tableau} in which every branch is closed is a
\emph{closed !!{tableau}} (for its set of assumptions). If !!a{tableau} is
not closed, i.e., if it contains at least one open branch, it is
\emph{open}.
\end{defn}

\begin{ex}
  Every set of assumptions on its own is !!a{tableau}, but it will
  generally not be closed. (Obviously, it is closed only if the
  assumptions already contain a pair of !!{signed formula}s
  $\sFmla{\True}{!A}$ and~$\sFmla{\False}{!A}$.)

  From !!a{tableau} (open or closed) we can obtain a new, larger one by
  applying one of the rules of inference to !!a{signed formula}~$!A$
  in it. The rule will append one or more !!{signed formula}s to the end of
  any branch containing the occurrence of~$!A$ to which we apply the
  rule.

  For instance, consider the assumption $\sFmla{\True}{!A \land \lnot
    !A}$. Here is the (open) !!{tableau} consisting of just that
  assumption:
  \begin{center}
    \begin{tableau}{}
      [\sFmla{\True}{\formula{A} \land \lnot \formula{A}}, just=\TAss]
    \end{tableau}{}
  \end{center}
  We obtain a new !!{tableau} from it by applying the $\TRule{\True}{\land}$
  rule to the assumption. That rule allows us to add two new lines to
  the !!{tableau}, $\sFmla{\True}{!A}$ and $\sFmla{\True}{\lnot !A}$:
  \begin{center}
    \begin{tableau}{}
      [\sFmla{\True}{\formula{A} \land \lnot \formula{A}}, just=\TAss
        [\sFmla{\True}{\formula{A}}, just={\TRule{\True}{\land}[1]},
          [\sFmla{\True}{\lnot \formula{A}}, just={\TRule{\True}{\land}[1]}
          ]
        ]
      ]
    \end{tableau}{}
  \end{center}
  When we write down !!{tableau}s, we record the rules we've applied
  on the right (e.g., $\TRule{\True}{\land} 1$ means that the
  !!{signed formula} on that line is the result of applying the
  $\TRule{\True}{\land}$ rule to the !!{signed formula} on line~$1$).
  This new !!{tableau} now contains additional !!{signed formula}s,
  but to only one ($\sFmla{\True}{\lnot !A}$) can we apply a rule (in
  this case, the $\TRule{\True}{\lnot}$ rule). This results in the closed
  !!{tableau}
  \begin{center}
    \begin{tableau}{}
      [\sFmla{\True}{\formula{A} \land \lnot \formula{A}}, just=\TAss
        [\sFmla{\True}{\formula{A}}, just={\TRule{\True}{\land}[1]}
          [\sFmla{\True}{\lnot \formula{A}}, just={\TRule{\True}{\land}[1]}
            [\sFmla{\False}{\formula{A}}, just={\TRule{\True}{\lnot}[3]}, close]
          ]
        ]
      ]
    \end{tableau}{}
  \end{center}
\end{ex}

\end{document}
