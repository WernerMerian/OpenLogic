% Part: first-order-logic
% Chapter: tableaux
% Section: provability-consistency

\documentclass[../../../include/open-logic-section]{subfiles}

\begin{document}

\iftag{FOL}
      {\olsection{fol}{tab}{prv}{\usetoken{S}{derivability} and Consistency}}
      {\olsection{pl}{tab}{prv}{\usetoken{S}{derivability} and Consistency}}

We will now establish a number of properties of the !!{derivability}
relation.  They are independently interesting, but each will play a
role in the proof of the completeness theorem.

\begin{prop}\ollabel{prop:provability-contr}
  If $\Gamma \Proves !A$ and $\Gamma \cup \{!A\}$ is
  inconsistent, then $\Gamma$ is inconsistent.
\end{prop}

\begin{proof}
There are finite $\Gamma_0 = \{!B_1, \dots, !B_n\}$ and $\Gamma_1
  =\{!C_1, \dots, !C_n\} \subseteq \Gamma$ such that
  \begin{align*}
    \{\sFmla{\False}{!A}, &
    \sFmla{\True}{!B_1}, \dots, \sFmla{\True}{!B_n}\} \\
    \{\sFmla{\True}{!A}, &
    \sFmla{\True}{!C_1}, \dots, \sFmla{\True}{!C_m}\}
  \end{align*}
  have closed !!{tableau}s.  Using the \Cut{} rule on $!A$ we can
  combine these into a single closed !!{tableau} that shows $\Gamma_0
  \cup \Gamma_1$ is inconsistent.  Since $\Gamma_0
  \subseteq \Gamma$ and $\Gamma_1 \subseteq \Gamma$, $\Gamma_0 \cup
  \Gamma_1 \subseteq \Gamma$, hence $\Gamma$~is inconsistent.
\end{proof}

\begin{prop}
\ollabel{prop:prov-incons}
$\Gamma \Proves !A$ iff $\Gamma \cup \{\lnot !A\}$ is inconsistent.
\end{prop}

\begin{proof}
First suppose $\Gamma \Proves !A$, i.e., there is
a closed !!{tableau} for
\[
\{\sFmla{\False}{!A},
\sFmla{\True}{!B_1}, \dots, \sFmla{\True}{!B_n}\}
\]
Using the $\TRule{\True}{\lnot}$ rule, this can be turned into a
closed !!{tableau} for
\[
\{\sFmla{\True}{\lnot !A},
\sFmla{\True}{!B_1}, \dots, \sFmla{\True}{!B_n}\}.
\]

On the other hand, if there is a closed !!{tableau} for the latter, we
can turn it into a closed !!{tableau} of the former by removing every
formula that results from \TRule{\True}{\lnot} applied to the first
assumption~$\sFmla{\True}{\lnot !A}$ as well as that assumption, and
adding the assumption $\sFmla{\False}{!A}$. For if a branch was closed
before because it contained the conclusion of \TRule{\True}{\lnot}
applied to $\sFmla{\True}{\lnot !A}$, i.e., $\sFmla{\False}{!A}$, the
corresponding branch in the new !!{tableau} is also closed. If a
branch in the old tableau was closed because it contained the
assumption $\sFmla{\True}{\lnot !A}$ as well as $\sFmla{\False}{\lnot
  !A}$ we can turn it into a closed branch by applying
$\TRule{\False}{\lnot}$ to $\sFmla{\False}{\lnot !A}$ to obtain
$\sFmla{\True}{!A}$. This closes the branch since we added
$\sFmla{\False}{!A}$ as an assumption.
\end{proof}

\begin{prob}
Prove that $\Gamma \Proves \lnot !A$ iff $\Gamma \cup \{!A\}$ is inconsistent.
\end{prob}

\begin{prop}\ollabel{prop:explicit-inc}
  If $\Gamma \Proves !A$ and $\lnot !A \in \Gamma$, then $\Gamma$ is
  inconsistent.
\end{prop}

\begin{proof}
  Suppose $\Gamma \Proves !A$ and $\lnot !A \in \Gamma$.  Then there
  are $!B_1$, \dots, $!B_n \in \Gamma$ such that \
  \[
  \{\sFmla{\False}{!A}, \sFmla{\True}{!B_1}, \dots, \sFmla{\True}{!B_n}\}
  \]
  has a closed tableau. Replace the assumption \sFmla{\False}{!A} by
  \sFmla{\True}{\lnot !A}, and insert the conclusion of
  \TRule{\True}{\lnot} applied to \sFmla{\False}{!A} after the
  assumptions. Any !!{sentence} in the !!{tableau} justified by appeal
  to line~$1$ in the old !!{tableau} is now justified by appeal to
  line~$n+1$. So if the old !!{tableau} was closed, the new one is.
  It shows that $\Gamma$ is inconsistent, since all assumptions are
  in~$\Gamma$.
\end{proof}

\begin{prop}\ollabel{prop:provability-exhaustive}
  If $\Gamma \cup \{!A\}$ and $\Gamma \cup \{\lnot !A\}$ are both
  inconsistent, then $\Gamma$ is inconsistent.
\end{prop}

\begin{proof}
  If there are $!B_1$, \dots, $!B_n \in \Gamma$ and $!C_1$, \dots,
  $!C_m \in \Gamma$ such that
  \begin{align*}
    \{\sFmla{\True}{!A}, &
    \sFmla{\True}{!B_1}, \dots, \sFmla{\True}{!B_n}\} \text{ and}\\
    \{\sFmla{\True}{\lnot !A}, &
    \sFmla{\True}{!C_1}, \dots, \sFmla{\True}{!C_m}\}
  \end{align*}
  both have closed !!{tableau}s, we can construct a single, combined
  !!{tableau} that shows that $\Gamma$ is inconsistent by using as
  assumptions $\sFmla{\True}{!B_1}$, \dots, $\sFmla{\True}{!B_n}$
  together with $\sFmla{\True}{!C_1}$, \dots, $\sFmla{\True}{!C_m}$,
  followed by an application of the \Cut{} rule. This yields two
  branches, one starting with $\sFmla{\True}{!A}$, the other with
  $\sFmla{\False}{!A}$.  
  
  On the left left side, add the part of the first
  !!{tableau} below its assumptions. Here, every rule application is still
  correct, since each of the assumptions of the first !!{tableau},
  including $\sFmla{\True}{!A}$, is available. Thus, every branch
  below $\sFmla{\True}{!A}$ closes. 
  
  On the right side, add the part of the second !!{tableau} below its
  assumption, with the results of any applications
  of~$\TRule{\True}{\lnot}$ to $\sFmla{\True}{\lnot !A}$ removed.  The
  conclusion of $\TRule{\True}{\lnot}$ to $\sFmla{\True}{\lnot !A}$ is
  $\sFmla{\False}{!A}$, which is nevertheless available, as it is the
  conclusion of the \Cut{} rule on the right side of the combined !!{tableau}.

  If a branch in the second tableau was closed because it contained
  the assumption $\sFmla{\True}{\lnot !A}$ (which no longer appears as
  an assumption in the combined !!{tableau}) as well as
  $\sFmla{\False}{\lnot !A}$, we can applying $\TRule{\False}{\lnot}$
  to $\sFmla{\False}{\lnot !A}$ to obtain $\sFmla{\True}{!A}$. Now the
  corresponding branch in the combined !!{tableau} also closes,
  because it contains the right-hand conclusion of the \Cut{} rule,
  $\sFmla{\False}{!A}$. If a branch in the second !!{tableau} closed
  for any other reason, the corresponding branch in the combined
  !!{tableau} also closes, since any !!{signed formula}s other than
  $\sFmla{\True}{\lnot !A}$ occurring on the branch in the old, second
  !!{tableau} also occur on the corresponding branch in the combined
  !!{tableau}.
  \end{proof}

\end{document}
