% Part: first-order-logic
% Chapter: tableaux
% Section: rules-and-proofs

\documentclass[../../../include/open-logic-section]{subfiles}

\begin{document}

\iftag{FOL}
      {\olsection{fol}{tab}{rul}{Rules and \usetoken{P}{tableau}}}
      {\olsection{pl}{tab}{rul}{Rules and \usetoken{P}{tableau}}}

!!^a{tableau} is a systematic survey of the possible ways
!!a{sentence} can be true or false in !!a{structure}. The building
blocks of a tableau are !!{signed formula}s: !!{sentence}s plus a
truth value ``sign,'' either $\True$ or~$\False$. These signed
!!{formula}s are arranged in a (downward growing) tree.

\begin{defn}
  A \emph{!!{signed formula}} is a pair consisting of a truth value
  and !!a{sentence}, i.e., either:
  \[
  \sFmla{\True}{!A} \text{ or } \sFmla{\False}{!A}.
  \]
\end{defn}

Intuitively, we might read $\sFmla{\True}{!A}$ as ``$!A$ might be
true'' and $\sFmla{\False}{!A}$ as ``$!A$ might be false'' (in some
!!{structure}).

Each !!{signed formula} in the tree is either an \emph{assumption}
(which are listed at the very top of the tree), or it is obtained from
!!a{signed formula} above it by one of a number of rules of
inference. There are two rules for each possible !!{main operator} of
the preceding !!{formula}, one for the case where the sign is~$\True$,
and one for the case where the sign is~$\False$. Some rules allow the
tree to branch, and some only add !!{signed formula}s to the branch.
A rule may be (and often must be) applied not to the immediately
preceding !!{signed formula}, but to any !!{signed formula} in the
branch from the root to the place the rule is applied.

A branch is \emph{closed} when it contains both $\sFmla{\True}{!A}$
and $\sFmla{\False}{!A}$. A closed !!{tableau} is one where every branch
is closed.  Under the intuitive interpretation, any branch describes a
joint possibility, but $\sFmla{\True}{!A}$ and $\sFmla{\False}{!A}$
are not jointly possible. In other words, if a branch is closed, the
possibility it describes has been ruled out. In particular, that means
that a closed !!{tableau} rules out all possibilities of simultaneously
making every assumption of the form $\sFmla{\True}{!A}$ true and every
assumption of the form~$\sFmla{\False}{!A}$ false.

A closed !!{tableau} \emph{for $!A$} is a closed !!{tableau} with
root~$\sFmla{\False}{!A}$. If such a closed !!{tableau} exists, all
possibilities for~$!A$ being false have been ruled out; i.e., $!A$
must be true in every !!{structure}.

\end{document}
