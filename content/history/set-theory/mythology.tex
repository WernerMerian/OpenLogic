\documentclass[../../../include/open-logic-section]{subfiles}

\begin{document}

\olsection{his}{set}{mythology}{More Myth than History?}

Looking back on these events with more than a century of hindsight, we
must be careful not to take these verdicts on trust. The results were
certainly novel, exciting, and surprising. But how truly shocking were
they? And did they really demonstrate that we should not rely on
geometric intuition?

On the question of shock, Gouv\^{e}a points out that Cantor's
famous note to Dedekind, ``\emph{je le vois, mais je ne le crois
pas}'' is taken rather out of context. Here is more of that context
(quoted from \cite{Gouvea2011}):
\begin{quote}
Please excuse my zeal for the subject if I make so many demands upon
your kindness and patience; the communications which I lately sent you
are even for me so unexpected, so new, that I can have no peace of
mind until I obtain from you, honoured friend, a decision about their
correctness. So long as you have not agreed with me, I can only say:
\emph{je le vois, mais je ne le crois pas.} 
\end{quote}
Cantor knew his result was ``so unexpected, so new''. But it is
doubtful that he ever found his result \emph{unbelievable}. As
Gouv\^{e}a points out, he was simply asking Dedekind to
check the proof he had offered. 

On the question of geometric intuition: Peano published his
space-filling curve without including any diagrams. But when Hilbert
published his curve, he explained his purpose: he would provide
readers with a clear way to understand Peano's result, if they ``help
themselves to the following geometric intuition''; whereupon he
included a series of \emph{diagrams} just like those provided in
\olref[his][set][pathology]{sec}. 

More generally: whilst diagrams have fallen rather out of fashion in
published proofs, there is no getting round the fact that
mathematicians \emph{frequently} use diagrams when proving things.
(Roughly put: good mathematicians know when they can rely upon
geometric intuition.)

In short: don't believe the hype; or at least, don't just take it on
trust. For more on this, you could read \cite{Giaquinto2007}.

\end{document}
