% Part: incompleteness
% Chapter: arithmetization-syntax
% Section: introduction

\documentclass[../../../include/open-logic-section]{subfiles}

\begin{document}

\olsection{inc}{art}{int}{Introduction}

In order to connect computability and logic, we need a way to talk
about the objects of logic (symbols, terms, !!{formula}s,
!!{derivation}s), operations on them, and their properties and
relations, in a way amenable to computational treatment.  We can do
this directly, by considering computable functions and relations on
symbols, sequences of symbols, and other objects built from them.
Since the objects of logical syntax are all finite and built from
!!a{enumerable} sets of symbols, this is possible for some models of
computation.  But other models of computation---such as the recursive
functions----are restricted to numbers, their relations and functions.
Moreover, ultimately we also want to be able to deal with syntax
within certain theories, specifically, in theories formulated in the
language of arithmetic.  In these cases it is necessary to
\emph{arithmetize} syntax, i.e., to represent syntactic objects,
operations on them, and their relations, as numbers, arithmetical
functions, and arithmetical relations, respectively. The idea, which
goes back to Leibniz, is to assign numbers to syntactic objects.

It is relatively straightforward to assign numbers to symbols as their
``codes.''  Some symbols pose a bit of a challenge, since, e.g., there
are infinitely many !!{variable}s, and even infinitely many
!!{function}s of each arity~$n$. But of course it's possible to assign
numbers to symbols systematically in such a way that, say, $\Obj v_2$
and $\Obj v_3$ are assigned different codes. Sequences of symbols
(such as terms and !!{formula}s) are a bigger challenge. But if we can
deal with sequences of numbers purely arithmetically (e.g., by the
powers-of-primes coding of sequences), we can extend the coding of
individual symbols to coding of sequences of symbols, and then further
to sequences or other arrangements of !!{formula}s, such as
!!{derivation}s. This extended coding is called ``G\"odel numbering.''
Every term, !!{formula}, and !!{derivation} is assigned a G\"odel
number.

By coding sequences of symbols as sequences of their codes, and by
chosing a system of coding sequences that can be dealt with using
computable functions, we can then also deal with G\"odel numbers using
computable functions.  In practice, all the relevant functions will be
primitive recursive.  For instance, computing the length of a sequence
and computing the $i$-th element of a sequence from the code of the
sequence are both primitive recursive. If the number coding the
sequence is, e.g., the G\"odel number of !!a{formula}~$!A$, we
immediately see that the length of !!a{formula} and the (code of the)
$i$-th symbol in !!a{formula} can also be computed from the G\"odel
number of~$!A$. It is a bit harder to prove that, e.g., the property
of being the G\"odel number of a correctly formed term or of a correct
!!{derivation} is primitive recursive.  It is nevertheless possible,
because the sequences of interest (terms, !!{formula}s,
!!{derivation}s) are inductively defined.

As an example, consider the operation of substitution. If $!A$ is a
formula, $x$ a variable, and $t$ a term, then $\Subst{!A}{t}{x}$ is
the result of replacing every free occurrence of~$x$ in~$!A$ by~$t$.
Now suppose we have assigned G\"odel numbers to $!A$, $x$, $t$---say,
$k$, $l$, and $m$, respectively.  The same scheme assigns a G\"odel
number to $\Subst{!A}{t}{x}$, say,~$n$.  This mapping---of $k$, $l$,
and $m$ to $n$---is the arithmetical analog of the substitution
operation. When the substitution operation maps $!A$, $x$, $t$ to
$\Subst{!A}{t}{x}$, the arithmetized substitution functions maps the
G\"odel numbers $k$, $l$, $m$ to the G\"odel number~$n$.  We will see
that this function is primitive recursive.

Arithmetization of syntax is not just of abstract interest, although
it was originally a non-trivial insight that languages like the
language of arithmetic, which do not come with mechanisms for
``talking about'' languages can, after all, formalize complex
properties of expressions.  It is then just a small step to ask what a
theory in this language, such as Peano arithmetic, can \emph{prove}
about its own language (including, e.g., whether !!{sentence}s are
provable or true).  This leads us to the famous limitative theorems of
G\"odel (about unprovability) and Tarski (the undefinability of
truth). But the trick of arithmetizing syntax is also important in
order to prove some important results in computability theory, e.g.,
about the computational power of theories or the relationship between
different models of computability.  The arithmetization of syntax
serves as a model for arithmetizing other objects and properties. For
instance, it is similarly possible to arithmetize configurations and
computations (say, of Turing machines). This makes it possible to
simulate computations in one model (e.g., Turing machines) in another
(e.g., recursive functions).

\end{document}
