% Part: incompleteness
% Chapter: arithmetization-syntax
% Section: proofs-in-lk

\documentclass[../../../include/open-logic-section]{subfiles}

\begin{document}

\olsection{inc}{art}{plk}{\usetoken{P}{derivation} in $\Log{LK}$}

\begin{explain}
In order to arithmetize !!{derivation}s, we must represent
!!{derivation}s as numbers. Since !!{derivation}s are trees of sequents
where each inference carries also a label, a recursive
representation is the most obvious approach: we represent a
!!{derivation} as a tuple, the components of which are the
end-sequent, the label, and the representations of the
sub-!!{derivation}s leading to the premises of the last inference.
\end{explain}

\begin{defn}
If $\Gamma$ is a finite sequence of !!{sentence}s, $\Gamma =
\tuple{!A_1, \dots, !A_n}$, then $\Gn{\Gamma} = \tuple{\Gn{!A_1},
  \dots, \Gn{!A_n}}$.

If $\Gamma \Sequent \Delta$ is a sequent, then a G\"odel number of
$\Gamma \Sequent \Delta$ is
\[
\Gn{\Gamma \Sequent \Delta} = \tuple{\Gn{\Gamma}, \Gn{\Delta}}
\]

If $\pi$ is !!a{derivation} in $\Log{LK}$, then $\Gn{\pi}$ is defined
as follows:
\begin{enumerate}
\item If $\pi$ consists only
  of the initial sequent $\Gamma \Sequent \Delta$, then $\Gn{\pi}$ is 
  \[
    \tuple{0, \Gn{\Gamma \Sequent \Delta}}.
  \]
\item If $\pi$ ends in an inference with one or two premises, has
  $\Gamma \Sequent \Delta$ as its conclusion, and $\pi_1$ and $\pi_2$ are the immediate subproof ending
  in the premise of the last inference, then $\Gn{\pi}$ is 
  \begin{align*}
    & \tuple{1, \Gn{\pi_1}, \Gn{\Gamma \Sequent \Delta}, k} \text{ or}\\ 
    & \tuple{2, \Gn{\pi_1}, \Gn{\pi_2}, \Gn{\Gamma \Sequent \Delta}, k}, 
  \end{align*}
  respectively, where $k$ is given by the following table according to
  which rule was used in the last inference:

  \begin{tabular}{lccccccc}
    \text{Rule:} & \LeftR{\Weakening} & \RightR{\Weakening} &
    \LeftR{\Contraction} & \RightR{\Contraction} &
    \LeftR{\Exchange} & \RightR{\Exchange} \\
    $k$: & 1 & 2 & 3 & 4 & 5 & 6 \\[2ex]
    \text{Rule:} & \LeftR{\lnot} & \RightR{\lnot} &
    \LeftR{\land} & \RightR{\land} & 
    \LeftR{\lor} & \RightR{\lor} \\
    $k$: & 7 & 8 & 9 & 10 & 11 & 12 \\[2ex]
    \text{Rule:} & \LeftR{\lif} & \RightR{\lif} &
    \LeftR{\lforall} & \RightR{\lforall} &
    \LeftR{\lexists} & \RightR{\lexists} \\
    $k$: & 13 & 14 & 15 & 16 & 17 & 18 \\[2ex]
    \text{Rule:} & \Cut & = \\
    $k$: & 19 & 20
  \end{tabular}
\end{enumerate}
\end{defn}

\begin{ex}
  Consider the very simple !!{derivation}
  \begin{prooftree}
    \Axiom$!A \fCenter !A$
    \RightLabel{\LeftR{\land}}
    \UnaryInf$!A \land !B \fCenter !A$
    \RightLabel{\RightR{\lif}}
    \UnaryInf$\fCenter (!A \land !B) \lif !A$
  \end{prooftree}
  The G\"odel number of the initial sequent would be $p_0 = \tuple{0,
    \Gn{!A \Sequent !A}}$.  The G\"odel number of the !!{derivation}
    ending in the conclusion of~$\LeftR{\land}$ would be $p_1 =
    \tuple{1, p_0, \Gn{!A \land !B \Sequent !A}, 9}$ ($1$ since
    $\LeftR{\land}$ has one premise, the G\"odel number of the
    conclusion~$!A \land !B \Sequent !A$, and $9$ is the number coding
    $\LeftR{\land}$). The G\"odel number of the entire !!{derivation} then
    is $\tuple{1, p_1, \Gn{\Sequent (!A \land !B) \lif !A)}, 14}$, i.e.,
  \[
  \tuple{1, \tuple{1, \tuple{0, \Gn{!A \Sequent !A)}}, \Gn{!A \land !B \Sequent !A}, 9},
    \Gn{\Sequent (!A \land !B) \lif !A}, 14}.
  \]
\end{ex}

\begin{explain}
Having settled on a representation of !!{derivation}s, we must also
show that we can manipulate such derivations primitive recursively,
and express their essential properties and relations so.  Some
operations are simple: e.g., given a G\"odel number~$p$ of
!!a{derivation}, $\fn{EndSeq}(p) = (p)_{(p)_0+1}$ gives us the G\"odel
number of its end-sequent and $\fn{LastRule}(p) = (p)_{(p)_0+2}$ the
code of its last rule.  The property $\fn{Sequent}(s)$ defined by
\[
  \len{s} = 2 \land \bforall{i<\len{(s)_0} + \len{(s)_1}}{\fn{Sent}(((s)_0 \concat (s)_1)_i)}
\]
holds of $s$ iff $s$ is the G\"odel number of a sequent consisting of
!!{sentence}s. Some are much harder.  We'll at least sketch how to do
this.  The goal is to show that the relation ``$\pi$ is
!!a{derivation} of~$!A$ from~$\Gamma$'' is a primitive recursive
relation of the G\"odel numbers of $\pi$ and~$!A$.
\end{explain}

\begin{prop}\ollabel{prop:followsby}
  The property $\fn{Correct}(p)$ which holds iff the last inference in
  the !!{derivation}~$\pi$ with G\"odel number~$p$ is correct, is
  primitive recursive.
\end{prop}

\begin{proof}
  $\Gamma \Sequent \Delta$ is an initial sequent if either there
  is !!a{sentence}~$!A$ such that $\Gamma \Sequent \Delta$ is $!A
  \Sequent !A$, or there is a term~$t$ such that $\Gamma \Sequent
  \Delta$ is $\emptyset \Sequent \eq[t][t]$.  In terms of G\"odel
  numbers, $\fn{InitSeq}(s)$ holds iff
  \begin{align*}
  \bexists{x < s}{} (\fn{Sent}(x) & \land
  s = \tuple{\tuple{x},\tuple{x}})
  \lor {}\\
  \bexists{t<s}{} (\fn{Term}(t) & \land
  s = \tuple{0, \tuple{\Gn{{\eq}(} \concat t \concat \Gn{,} \concat t \concat \Gn{)}}}).
  \end{align*}

  We also have to show that for each rule of inference~$R$ the
  relation $\fn{FollowsBy}_R(p)$ is primitive recursive, where
  $\fn{FollowsBy}_R(p)$ holds iff $p$ is the G\"odel number of
  !!{derivation}~$\pi$, and the end-sequent of~$\pi$ follows
  by a correct application of~$R$ from the immediate
  sub-!!{derivation}s of~$\pi$.

  A simple case is that of the \RightR{\land} rule. If $\pi$ ends in
  a correct $\RightR{\land}$ inference, it looks like this:
  \begin{prooftree}
    \AxiomC{}
    \RightLabel{$\pi_1$}
    \Deduce$\Gamma \fCenter \Delta, !A$
    
    \AxiomC{}
    \RightLabel{$\pi_2$}
    \Deduce$\Gamma \fCenter \Delta, !B$

    \RightLabel{\RightR\land}
    \BinaryInf$\Gamma \fCenter \Delta, !A \land !B$
  \end{prooftree}
  So, the last inference in the !!{derivation} $\pi$
  is a correct application of $\RightR{\land}$ iff
  there are sequences of !!{sentence}s $\Gamma$ and $\Delta$
  as well as two !!{sentence}s $!A$ and~$!B$ such that
  the end-sequent of $\pi_1$ is $\Gamma \Sequent \Delta, !A$,
  the end-sequent of $\pi_2$ is $\Gamma \Sequent \Delta, !B$,
  and the end-sequent of $\pi$ is $\Gamma \Sequent \Delta, !A \land !B$.
  We just have to translate this into G\"odel
  numbers.  If $s = \Gn{\Gamma \Sequent \Delta}$ then $(s)_0 =
  \Gn{\Gamma}$ and $(s)_1 = \Gn{\Delta}$.  So,
  $\fn{FollowsBy}_{\RightR{\land}}(p)$ holds iff
\begin{align*}
& \bexists{g < p}{\bexists{d < p}{\bexists{a < p}{\bexists{b < p}{\quad}}}} \\
& \qquad \fn{EndSequent}(p) = 
  \tuple{g, d \concat 
    \tuple{\Gn{(} \concat a \concat \Gn{\land} \concat b \concat \Gn{)}}} \land {} \\
& \qquad \fn{EndSequent}((p)_1) = \tuple{g, d \concat \tuple{a}} \land {}\\
& \qquad \fn{EndSequent}((p)_2) = \tuple{g, d \concat \tuple{b}} \land {}\\
& \qquad (p)_0 = 2 \land \fn{LastRule}(p) = 10.
\end{align*}
The individual lines express, respectively, ``there is a
sequence~($\Gamma$) with G\"odel number~$g$, there is a
sequence~($\Delta$) with G\"odel number~$d$, !!a{formula}~($!A$) with
G\"odel number~$a$, and !!a{formula}~($!B$) with G\"odel number~$b$,''
such that ``the end-sequent of $\pi$ is $\Gamma \Sequent \Delta, !A
\land !B$,'' ``the end-sequent of $\pi_1$ is $\Gamma \Sequent \Delta,
!A$,'' ``the end-sequent of $\pi_2$ is $\Gamma \Sequent \Delta, !B$,''
and ``$\pi$ has two immediate subderivations and the last inference
rule is $\RightR\land$ (with number~$10$).''

The last inference in~$\pi$ is a correct application of
$\RightR\lexists$ iff there are sequences $\Gamma$ and $\Delta$,
!!a{formula}~$!A$, a variable~$x$, and a term~$t$, such that
the end-sequent of $\pi$ is $\Gamma \Sequent \Delta, \lexists[x][!A]$
and the end-sequent of $\pi_1$ is $\Gamma \Sequent \Delta,
\Subst{!A}{t}{x}$. So in terms of G\"odel numbers, we have
$\fn{FollowsBy}_{\RightR\lexists}(p)$
iff
\begin{align*}
  & \bexists{g<p}{\bexists{d<p}{\bexists{a<p}{\bexists{x<p}{\bexists{t<p}{\quad}}}}}\\
  & \qquad \fn{EndSequent}(p) = 
  \tuple{
    g,
    d \concat \tuple{\Gn{\lexists} \concat x \concat a}
  } \land {}\\
  & \qquad \fn{EndSequent}((p)_1) = 
  \tuple{
    g,
    d \concat \tuple{\fn{Subst}(a, t, x)}
  } \land {}\\
  & \qquad (p)_0 = 1 \land \fn{LastRule}(p) = 18.
\end{align*}
We then define $\fn{Correct}(p)$ as
\begin{multline*}
  \fn{Sequent}(\fn{EndSequent}(p)) \land {}\\ 
  [(\fn{LastRule}(p) = 1 \land 
  \fn{FollowsBy}_{\LeftR\Weakening}(p)) \lor \dots \lor {}\\
  (\fn{LastRule}(p) = 20 \land \fn{FollowsBy}_{\eq}(p)) \lor {}\\
  (p)_0 = 0 \land \fn{InitialSeq}(\fn{EndSequent}(p))]
\end{multline*}
The first line ensures that the end-sequent of~$d$ is actually a
sequent consisting of !!{sentence}s. The last line covers the case
where $p$ is just an initial sequent.
\end{proof}

\begin{prob}
  Define the following properties as in
  \olref[inc][art][plk]{prop:followsby}:
  \begin{enumerate}
  \item $\fn{FollowsBy}_{\Cut}(p)$,
  \item $\fn{FollowsBy}_{\LeftR\lif}(p)$,
  \item $\fn{FollowsBy}_{\eq}(p)$,
  \item $\fn{FollowsBy}_{\RightR{\lforall}}(p)$.
  \end{enumerate}
  For the last one, you will have to also show that you can test
  primitive recursively if the last inference of the
  !!{derivation} with G\"odel number~$p$ satisfies the eigenvariable
  condition, i.e., the eigenvariable~$a$ of the $\RightR{\lforall}$
  does not occur in the end-sequent.
  \end{prob}
  

\begin{prop}
  \ollabel{prop:deriv}
  The relation $\fn{Deriv}(p)$ which holds if $p$ is the G\"odel
  number of a correct !!{derivation}~$\pi$, is primitive recursive.
\end{prop}

\begin{proof}
  !!^a{derivation}~$\pi$ is correct if every one of its inferences
  is a correct application of a rule, i.e., if every one of its
  sub-!!{derivation}s ends in a correct inference. So, $\fn{Deriv}(d)$
  iff
  \[
  \bforall{i<\len{\fn{SubtreeSeq}(p)}}{\fn{Correct}((\fn{SubtreeSeq}(p))_i}.
  \]
\end{proof}


\begin{prop}
Suppose $\Gamma$ is a primitive recursive set of !!{sentence}s.  Then
the relation $\Prf[\Gamma](x, y)$ expressing ``$x$ is the code of
!!a{derivation}~$\pi$ of $\Gamma_0 \Sequent !A$ for some finite
$\Gamma_0 \subseteq \Gamma$ and $y$ is the G\"odel number of~$!A$'' is
primitive recursive.
\end{prop}

\begin{proof}
Suppose ``$y \in \Gamma$'' is given by the primitive recursive
predicate~$R_\Gamma(y)$.  We have to show that $\Prf[\Gamma](x, y)$
which holds iff $y$ is the G\"odel number of a sentence~$!A$ and
$x$~is the code of an $\Log{LK}$-!!{derivation} with end-sequent
$\Gamma_0 \Sequent !A$ is primitive recursive.

By the previous proposition, the property $\fn{Deriv}(x)$ which holds
iff $x$ is the code of a correct derivation~$\pi$ in $\Log{LK}$ is
primitive recursive.  If $x$ is such a code, then $\fn{EndSequent}(x)$
is the code of the end-sequent of~$\pi$, and so
$(\fn{EndSequent}(x))_0$ is the code of the left side of the end
sequent and $(\fn{EndSequent}(x))_1$ the right side.  So we can
express ``the right side of the end-sequent of~$\pi$ is~$!A$'' as
$\len{(\fn{EndSequent}(x))_1} = 1 \land ((\fn{EndSequent}(x))_1)_0 =
x$.  The left side of the end-sequent of $\pi$ is of course
automatically finite, we just have to express that every sentence in
it is in~$\Gamma$.  Thus we can define $\Prf[\Gamma](x, y)$ by
\begin{align*}
\Prf[\Gamma](x, y) \defiff {}&
\fn{Deriv}(x) \land {} \\
& \bforall{i <
  \len{(\fn{EndSequent}(x))_0}}{R_\Gamma(((\fn{EndSequent}(x))_0)_i)} \land {}\\
& \len{(\fn{EndSequent}(x))_1} = 1 \land ((\fn{EndSequent}(x))_1)_0 = y.
\end{align*}
\end{proof}

\end{document}
