% Part: incompleteness
% Chapter: incompleteness-provability
% Section: introduction

\documentclass[../../../include/open-logic-section]{subfiles}

\begin{document}

\olsection{inc}{inp}{int}{Introduction}

Hilbert thought that a system of axioms for a mathematical structure,
such as the natural numbers, is inadequate unless it allows one to
derive all true statements about the structure. Combined with his
later interest in formal systems of deduction, this suggests that he
thought that we should guarantee that, say, the formal systems we are
using to reason about the natural numbers is not only consistent, but
also \emph{complete}, i.e., every statement in its language is either
!!{derivable} or its negation is. G\"odel's first incompleteness theorem
shows that no such system of axioms exists: there is no complete,
consistent, !!{axiomatizable} formal system for arithmetic.  In fact,
no ``sufficiently strong,'' consistent, !!{axiomatizable} mathematical
theory is complete.

A more important goal of Hilbert's, the centerpiece of his program for
the justification of modern (``classical'') mathematics, was to find
finitary consistency proofs for formal systems representing classical
reasoning.  With regard to Hilbert's program, then, G\"odel's second
incompleteness theorem was a much bigger blow. The second
incompleteness theorem can be stated in vague terms, like the first
incompleteness theorem. Roughly speaking, it says that no sufficiently
strong theory of arithmetic can prove its own consistency. We will
have to take ``sufficiently strong'' to include a little bit more
than~$\Th{Q}$.

The idea behind G\"odel's original proof of the incompleteness theorem
can be found in the Epimenides paradox. Epimenides, a Cretan, asserted
that all Cretans are liars; a more direct form of the paradox is the
assertion ``this sentence is false.'' Essentially, by replacing truth
with !!{derivability}, G\"odel was able to formalize !!a{sentence}
which, in a roundabout way, asserts that it itself is not
!!{derivable}.  If that !!{sentence} were !!{derivable}, the theory
would then be inconsistent.  G\"odel showed that the negation of that
!!{sentence} is also not !!{derivable} from the system of axioms he was
considering. (For this second part, G\"odel had to assume that the
theory~$\Th{T}$ is what's called ``$\omega$-consistent.''
$\omega$-Consistency is related to consistency, but is a stronger
property.\footnote{That is, any $\omega$-consistent theory is
consistent, but not vice versa.} A few years after G\"odel, Rosser
showed that assuming simple consistency of~$\Th{T}$ is enough.)

The first challenge is to understand how one can construct !!a{sentence}
that refers to itself. For every !!{formula}~$!A$ in the language
of~$\Th{Q}$, let~$\gn{!A}$ denote the numeral corresponding
to~$\Gn{!A}$. Think about what this means: $!A$~is !!a{formula} in the
language of~$\Th{Q}$, $\Gn{!A}$~is a natural number, and $\gn{!A}$~is
a \emph{term} in the language of~$\Th{Q}$. So every !!{formula}~$!A$
in the language of~$\Th{Q}$ has a \emph{name},~$\gn{!A}$, which is a
term in the language of~$\Th{Q}$; this provides us with a conceptual
framework in which !!{formula}s in the language of~$\Th{Q}$ can ``say''
things about other !!{formula}s. The following lemma is known as the
fixed-point lemma.

\begin{lem}
Let $\Th{T}$ be any theory extending~$\Th{Q}$, and let $!B(x)$ be any
!!{formula} with only the variable~$x$ free. Then there is
!!a{sentence}~$!A$ such that $\Th{T} \Proves!A \liff !B(\gn{!A})$.
\end{lem}

The lemma asserts that given any property $!B(x)$, there is
!!a{sentence}~$!A$ that asserts ``$!B(x)$ is true of me,'' and
$\Th{T}$ ``knows'' this.

How can we construct such !!a{sentence}? Consider the following version
of the Epimenides paradox, due to Quine:
\begin{quote}
``Yields falsehood when preceded by its quotation'' yields falsehood
when preceded by its quotation.
\end{quote}
This sentence is not directly self-referential. It simply makes an
assertion about the syntactic objects between quotes, and, in doing
so, it is on par with sentences like
\begin{enumerate}
\item ``Robert'' is a nice name.
\item ``I ran.'' is a short sentence.
\item ``Has three words'' has three words.
\end{enumerate}
But what happens when one takes the phrase ``yields falsehood when
preceded by its quotation,'' and precedes it with a quoted version of
itself? Then one has the original sentence!{} In short, the sentence
asserts that it is false.

\end{document}
