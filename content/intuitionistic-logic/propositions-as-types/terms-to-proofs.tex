% part: intuitionistic-logic
% chapter: propositions-as-types
% section: terms-to-proofs

\documentclass[../../../include/open-logic-section]{subfiles}

\begin{document}

\olsection{int}{pty}{tp}{Recovering \usetoken{P}{derivation} from Proof Terms}

Now let us consider the other direction: translating terms back to
natural deduction trees. We will use still use the double refutation of 
the excluded middle as example, and let $S$ denote this term, i.e.,
\[
\lambd[\typeof{y}{(!A \lor (!A \lif \lfalse))\lif \lfalse}][y
    (\ori{2}{!A}{\lambd[\typeof{x}{!A}][y \ori{1}{!A \lif \lfalse}{x}]})] :
  ((!A \lor (!A \lif \lfalse))\lif \lfalse)\lif \lfalse
\]

For each natural deduction rule, the term in the conclusion is always
formed by wrapping some operator around the terms assigned to the
premise(s). Rules correspond uniquely to such operators. For example,
from the structure of the $S$ we infer that the last rule applied must
be $\Intro{\lif}$, since it is of the form
$\lambd[\typeof{y}{\dots}][\dots]$, and the $\lambd$ operator
corresponds to $\Intro\lif$. In general we can recover the skeleton of
the !!{derivation} solely by the structure of the term, e.g.,
\begin{gather*}
  \AxiomC{$\Discharge{y: }{2}$}
  \AxiomC{$\Discharge{y: }{2}$}
  \AxiomC{$\Discharge{x}{1}$}
  \RightLabel{$\Intro\lor_1$}
  \UnaryInfC{$\ori{1}{!A \lif \lfalse}{x} : $}
  \RightLabel{\Elim\lif}
  \BinaryInfC{$y (\ori{1}{!A \lif \lfalse}{x}) : $}
  \DischargeRule{\Intro\lif}{1}
  \UnaryInfC{$\lambd[\typeof{x}{!A}][y (\ori{1}{!A \lif \lfalse}{x})] : $}
  \RightLabel{$\Intro\lor_2$}
  \UnaryInfC{$\ori{2}{!A}{\lambd[\typeof{x}{!A}][y
        \ori{1}{!A \lif \lfalse}{x}]} : $}
  \RightLabel{\Elim\lif}
  \BinaryInfC{$y (\ori{2}{!A}{\lambd[\typeof{x}{!A}][y
        \ori{1}{!A \lif \lfalse}{x}]}): $}
  \DischargeRule{\Intro\lif}{2}
  \UnaryInfC{$\lambd[\typeof{y}{(!A \lor (!A \lif \lfalse))\lif \lfalse}][y
    (\ori{2}{!A}{\lambd[\typeof{x}{!A}][y (\ori{1}{!A \lif \lfalse}{x})]})]$ :}
  \DisplayProof
\end{gather*}

Our next step is to recover the !!{formula}s these terms witness. We
define a function~$F(\Gamma,M)$ which denotes the !!{formula}
witnessed by $M$ in context $\Gamma$, by induction on $M$ as follows:
\begin{align*}
  F(\Gamma, x) &= \Gamma(x) \\
  F(\Gamma, \andi{N_1}{N_2} & = F(\Gamma, N_1) \land F(\Gamma, N_2)\\
  F(\Gamma, \ande{i}{N} & = !A_i \text{ if } F(\Gamma, N) = !A_1 \land !A_2\\
  F(\Gamma, \ori{i}{!A}{N} & = \begin{cases} F(N) \lor !A & \text{if } i = 1\\
    !A \lor F(N) & \text{if } i = 2\end{cases}\\
  F(\Gamma, \ore{M}{x_1}{N_1}{x_2}{N_2}) &= F(\Gamma \cup
  \{x_i:F(\Gamma, M)\}, N_i) \\
  F(\Gamma, \lambd[\typeof{x}{!A}][N]) &= !A \lif F(\Gamma \cup
  \{x:!A\}, N) \\
  F(\Gamma, NM) & = !B \text{ if } F(\Gamma, N) = !A \lif !B
\end{align*}
where $\Gamma(x)$ means the formula mapped to by $x$ in $\Gamma$
and $\Gamma \cup \{x:!A\}$
is a context exactly as $\Gamma$ except mapping $x$ to $!A$,
whether or not $x$ is already in $\Gamma$.

Note there are cases where $F(\Gamma, M)$ is not defined, for example:
\begin{enumerate}
\item In the first line, it is possible that $x$ is not in $\Gamma$.
\item In recursive cases, the inner invocation may be undefined, making
  the outer one undefined too.
\item In the third line, its only defined when $F(\Gamma, M)$ is of
  the form $!A_1 \lor !A_2$, and the right hand is independent on $i$.
\end{enumerate}

As we recursively compute $F(\Gamma, M)$, we work our way up the
natural deduction !!{derivation}. The every step in the computation of
$F(\Gamma, M)$ corresponds to a term in the !!{derivation} to which
the !!{derivation}-to-term translation assigns~$M$, and the formula
computed is the end-!!{formula} of the derivation. However, the result
may not be defined for some choices of~$\Gamma$. We say that such
pairs $\tuple{\Gamma, M}$ are \emph{ill-typed}, and otherwise
\emph{well-typed}. However, if the term $M$ results from translating
!!a{derivation}, and the !!{formula}s in $\Gamma$ correspond to the
!!{undischarged} assumptions of the !!{derivation}, the pair
$\tuple{\Gamma, M}$ will be well-typed.

\begin{prop}
  If $D$ is !!a{derivation} with !!{undischarged} assumptions~$!A_1$,
  \dots, $!A_n$, $M$ is the proof term associated with~$D$ and $\Gamma
  = \{x_1:!A_1, \dots, x_n:!A_n\}$, then the result of recovering
  !!{derivation} from $M$ in context~$\Gamma$ is $D$.
\end{prop}

In the other direction, if we first translate a typing pair to natural
deduction and then translate it back, we won't get the same pair back
since the choice of variables for the !!{undischarged} assumptions is
underdetermined. For example, consider the pair $\tuple{\{x:!A, y:!A
  \lif !B\}, y x}$. The corresponding !!{derivation} is
\begin{prooftree}
  \AxiomC{$!A \lif !B$}
  \AxiomC{$!A$}
  \RightLabel{\Elim\lif}
  \BinaryInfC{$!B$}
\end{prooftree}
By assigning different variables to the !!{undischarged} assumptions,
say, $u$ to $!A \lif !B$ and $v$ to $!A$, we would get the term $uv$
rather than~$yx$.  There is a connection, though: the terms will be
the same up to renaming of variables.

Now we have established the correspondence between typing pairs and
natural deduction, we can prove theorems for typing pairs and transfer
the result to natural deduction !!{derivation}s.

Similar to what we did in the natural deduction section, we can make
some observations here too. Let $\Gamma \Proves M:!A$ denote that
there is a pair $(\Gamma, M)$ witnessing the formula $!A$. Then always
$\Gamma \Proves x:!A$ if $x:!A \in \Gamma$, and the following rules
are valid:

\begin{gather*}
  \AxiomC{$\Gamma \Proves M_1:!A_1$}
  \AxiomC{$\Delta \Proves M_2:!A_2$}
  \RightLabel{$\Intro{\land}$}
  \BinaryInfC{$\Gamma,\Delta \Proves \tuple{M_1,M_2} : !A_1 \land !A_2 $}
  \DisplayProof
  \quad
  \AxiomC{$\Gamma \Proves M:!A_1 \land !A_2$}
  \RightLabel{$\Elim{\land}_i$}
  \UnaryInfC{$\Gamma \Proves \ande{i}{M} : !A_i$}
  \DisplayProof
  \\
  \AxiomC{$\Gamma \Proves M_1: !A_1$}
  \RightLabel{$\Intro{\lor}_1$}
  \UnaryInfC{$\Gamma \Proves \ori{1}{!A_2}M :  !A_1 \lor !A_2$}
  \DisplayProof
  \quad
  \AxiomC{$\Gamma \Proves M_2: !A_2$}
  \RightLabel{$\Intro{\lor}_2$}
  \UnaryInfC{$\Gamma \Proves \ori{2}{!A_1}M :  !A_1 \lor !A_2$}
  \DisplayProof
  \\
  \AxiomC{$\Gamma \Proves M : !A \lor !B$}
  \AxiomC{$\Delta_1, x_1: !A_1 \Proves N_1 : !C$}
  \AxiomC{$\Delta_2, x_2: !A_2 \Proves N_2 : !C$}
  \RightLabel{$\Elim{\lor}$}
  \TrinaryInfC{$\Gamma, \Delta, \Delta' \Proves \ore{M}{x_1}{N_1}{x_2}{N_2} : !C$}
  \DisplayProof
  \\
  \AxiomC{$\Gamma, x : !A \Proves N : !B$}
  \RightLabel{\Intro{\lif}}
  \UnaryInfC{$\Gamma \Proves \lambd[\typeof{x}{!A}][N] : !A \lif !B$}
  \DisplayProof
  \quad
  \AxiomC{$\Gamma \Proves Q : !A$}
  \AxiomC{$\Delta \Proves P : !A \lif !B$}
  \RightLabel{\Elim{\lif}}
  \BinaryInfC{$\Gamma, \Delta \Proves P Q : !B$}
  \DisplayProof
  \\
  \AxiomC{$\Gamma \Proves M : \lfalse$}
  \RightLabel{\Elim{\lfalse}}
  \UnaryInfC{$\Gamma  \Proves \falsee{!A}{M} : !A$}
  \DisplayProof
\end{gather*}

These are the typing rules of the simply typed lambda calculus
extended with product, sum and bottom.

In addition, the $F(\Gamma, M)$ is actually a type checking algorithm;
it returns the type of the term with respect to the context, or is
undefined if the term is ill-typed with respect to the context.

\end{document}
