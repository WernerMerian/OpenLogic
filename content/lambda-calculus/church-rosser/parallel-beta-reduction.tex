% Part: lambda-calculus
% Chapter: church-rosser
% Section: parallel-beta-reduction

\documentclass[../../../include/open-logic-section]{subfiles}

\begin{document}

\olsection{lam}{cr}{pb}{Parallel $\beta$-reduction}

We introduce the notion of \emph{parallel $\beta$-reduction}, and
prove the it has the Church--Rosser property.

\begin{defn}[parallel $\beta$-reduction, $\bredpar$] \ollabel{defn:bredpar}
  Parallel reduction ($\bredpar$) of terms is inductively defined as follows:
  \begin{enumerate}
    \item \ollabel{defn:bredpar1} $x \bredpar x$.
    \item \ollabel{defn:bredpar2} If $N \xrightarrow{\beta} N'$ then $\lambd[x][N] \bredpar
      \lambd[x][N']$.
    \item \ollabel{defn:bredpar3} If $P \bredpar P'$ and $Q \bredpar Q'$ then $PQ \bredpar
      P'Q'$.
    \item \ollabel{defn:bredpar4} If $N \bredpar N'$ and $Q \bredpar Q'$ then
      $(\lambd[x][N])Q \bredpar \Subst{N'}{Q'}{x}$.
  \end{enumerate}
\end{defn}

Parallel $\beta$-reduction allows us to reduce any number of redices in a
term in one step. It is different from $\beta$-reduction in the sense that we
can only contract redices that occur in the original term, but not redices
arising from parallel $\beta$-reduction. For example, the term
$(\lambd[f][fx])(\lambd[y][y])$ can only be parallel $\beta$-reduced
to itself or to $(\lambd[y][y])x$, but not further to~$x$, although it
$\beta$-reduces to~$x$, because this redex arises only after one step 
of parallel $\beta$-reduction. A second parallel $\beta$-reduction
step yields~$x$, though.

\begin{thm}\ollabel{thm:refl}
  $M \bredpar M$.
\end{thm}

\begin{proof}
  Exercise.
\end{proof}

\begin{prob}
  Prove \olref[lam][cr][pb]{thm:refl}.
\end{prob}

\begin{defn}[$\beta$-complete development]\ollabel{defn:bcd}
  The \emph{$\beta$-complete development}~$\bcd{M}$ of $M$ is defined
  inductively as follows:
  \begin{align}
    \bcd{x} &= x \ollabel{defn:bcd1} \\
    \bcd{(\lambd[x][N])} &= \lambd[x][\bcd{N}] \ollabel{defn:bcd2}\\
    \bcd{(PQ)} &= \bcd{P}\bcd{Q} && \text{if $P$ is not a $\lambd$-abstract} 
    \ollabel{defn:bcd3} \\
    \bcd{((\lambd[x][N])Q)} &= \Subst{\bcd{N}}{\bcd{Q}}{x} \ollabel{defn:bcd4}
  \end{align}
\end{defn}

The $\beta$-complete development of a term, as its name suggests, is a
``complete parallel reduction.'' While for parallel $\beta$-reduction we still
can choose to not contract a redex, for complete development we have
no choice but to contract all of them. Thus the complete development
of $(\lambd[f][fx])(\lambd[y][y])$ is $(\lambd[y][y])x$, not itself.

\begin{editorial}
  This definition has the problem that we haven't introduced how to
  define functions on ($\lambd$-)terms recursively. Will fix in future.
\end{editorial}

\begin{lem}\ollabel{lem:comp}
  If $M \bredpar M'$ and $R \bredpar R'$, then $\Subst{M}{R}{y}
  \bredpar \Subst{M'}{R'}{y}$.
\end{lem}

\begin{proof}
  By induction on the derivation of $M \bredpar M'$.
  \begin{enumerate}
    \item The last step is \olref{defn:bredpar1}: Exercise.
    \item The last step is \olref{defn:bredpar2}: Then $M$ is 
      $\lambd[x][N]$ and $M'$ is $\lambd[x][N']$,
      where $N \bredpar N'$. We want to prove that
      $\Subst{(\lambd[x][N])}{R}{y} \bredpar
      \Subst{(\lambd[x][N'])}{R'}{y}$, i.e.,
      $\lambd[x][\Subst{N}{R}{y}] \bredpar
      \lambd[x][\Subst{N'}{R}{y}]$. This follows immediately by
      \olref{defn:bredpar2} and the induction hypothesis.
    \item The last step is \olref{defn:bredpar3}: Exercise.
    \item The last step is \olref{defn:bredpar4}: $M$ is
      $(\lambd[x][N])Q$ and $M'$ is $\Subst{N'}{Q'}{x}$. We want to
      prove that $\Subst{((\lambd[x][N])Q)}{R}{y} \bredpar
      \Subst{\Subst{N'}{Q'}{x}}{R'}{y}$, i.e.,
      $(\lambd[x][\Subst{N}{R}{y}])\Subst{Q}{R}{y} \bredpar
      \Subst{\Subst{N'}{R'}{y}}{\Subst{Q'}{R'}{y}}{x}$. This follows
      by \olref{defn:bredpar4} and the induction hypothesis.
  \end{enumerate}
\end{proof}

\begin{prob}
  Complete the proof of \olref[lam][cr][pb]{lem:comp}.
\end{prob}

\begin{lem}\ollabel{lem:cont}
  If $M \bredpar M'$ then $M' \bredpar \bcd{M}$.
\end{lem}
\begin{proof}
  By induction on the derivation of $M \bredpar M'$.
  \begin{enumerate}
    \item The last rule is \olref{defn:bredpar1}: Exercise.
    \item The last rule is \olref{defn:bredpar2}: $M$ is 
      $\lambd[x][N]$ and $M'$ is $\lambd[x][N']$ with
      $N \bredpar N'$. We want to show that $\lambd[x][N'] \bredpar
      \bcd{(\lambd[x][N])}$, i.e., $\lambd[x][N'] \bredpar
      \lambd[x][\bcd{N}]$ by \olref{defn:bcd2}. It follows by
      \olref{defn:bredpar2} and the induction hypothesis.
    \item The last rule is \olref{defn:bredpar3}:$M$ is $PQ$ and $M'$
      is $P'Q'$ for some $P$, $Q$, $P'$ and $Q'$, with $P \bredpar P'$
      and $Q \bredpar Q'$. By induction hypothesis, we have $P'
      \bredpar \bcd{P}$ and $Q' \bredpar \bcd{Q}$.
      \begin{enumerate}
        \item If $P$ is $\lambd[x][N]$ for some $x$ and $N$, then
          $P'$ must be $\lambd[x][N']$ for some $N'$ with 
          $N \bredpar N'$. By induction hypothesis we have $N' \bredpar \bcd{N}$ and
          $Q' \bredpar \bcd{Q}$. Then $(\lambd[x][N'])Q' \bredpar
          \Subst{\bcd{N}}{\bcd{Q}}{x}$ by \olref{defn:bredpar4}.
        \item If $P$ is not a $\lambd$-abstract, then $P'Q' \bredpar
          \bcd{P}\bcd{Q}$ by \olref{defn:bredpar3}, and the right-hand
          side is $\bcd{PQ}$ by \olref{defn:bcd3}.
      \end{enumerate}
    \item The last rule is \olref{defn:bredpar4}: $M$ is
      $(\lambd[x][N])Q$  and $M'$ is $\Subst{N'}{Q'}{x}$ for some $x$,
      $N$, $Q$, $N'$, and~$Q'$, with $N \bredpar N'$ and $Q \bredpar
      Q'$. By induction hypothesis we know $N' \bredpar \bcd{N}$ and
      $Q' \bredpar \bcd{Q}$. By \olref{lem:comp} we have
      $\Subst{N'}{Q'}{x} \bredpar \Subst{\bcd{N}}{\bcd{Q}}{x}$, the
      right-hand side of which is exactly $\bcd{((\lambd[x][N])Q)}$.
  \end{enumerate}
\end{proof}

\begin{prob}
  Complete the proof of \olref[lam][cr][pb]{lem:cont}.
\end{prob}

\begin{thm}\ollabel{thm:cr}
  $\bredpar$ has the Church--Rosser property.
\end{thm}

\begin{proof}
  Immediate from \olref{lem:cont}.
\end{proof}

\end{document}

