% Part: lambda-calculus
% Chapter: syntax
% Section: eta

\documentclass[../../../include/open-logic-section]{subfiles}

\begin{document}

\olsection{lam}{syn}{eta}{$\eta$-conversion}

There is another relation on $\lambda$ terms. In \olref[fv]{sec} we
used the example $\lambd[x][(fx)]$, which accepts an argument and
applies $f$ to it.  In other words, it is the same function as~$f$:
$\lambd[x][(fx)]N$ and $fN$ both reduce to~$fN$.  We use
$\eta$-reduction (and $\eta$-extension) to capture this idea.

\begin{defn}[$\eta$-contraction, $\eredone$]
  \ollabel{defn:beredone}
  \emph{$\eta$-contraction} ($\eredone$) is the smallest compatible relation
  on terms satisfying the following condition:
  \[
    \lambd[x][M x] \eredone M \text{ provided } x \notin FV(M)
  \]
\end{defn}

\begin{defn}[$\beta\eta$-reduction, $\bered$] \ollabel{defn:bered}
  \emph{$\beta\eta$-reduction} ($\bered$) is the smallest reflexive,
  transitive relation on terms containing $\bredone$ and $\eredone$,
  i.e., the rules of reflexivity and transitive plus the following two
  rules:
  \begin{enumerate}
  \item If $M \bredone N$ then $M \bered N$. \ollabel{defn:bered3}
  \item If $M \eredone N$ then $M \bered N$. \ollabel{defn:bered4}
  \end{enumerate}
    
\end{defn}

\begin{defn}
  We extend the equivalence relation $\equal$ with the $\eta$-conversion rule:
  \[
  \lambd[x][f x] \equal f
  \]
  and denote the extended relation as $\equal[\eta]$.
\end{defn}

$\eta$-equivalence is important because it is related to extensionality
of lambda terms:

\begin{defn}[Extensionality]
  We extend the equivalence relation $\equal$ with the (\ext) rule:
  \begin{center}
  If $Mx \equal Nx$ then $M \equal N$, provided $x \notin FV(MN)$.
  \end{center}
  and denote the extended relation as $\equal[\ext]$.
\end{defn}

Roughly speaking, the rule states that two terms, viewed as functions,
should be considered equal if they behave the same for the same
argument.

We now prove that the $\eta$ rule provides exactly the extensionality,
and nothing else.

\begin{thm}
  $M \equal[\ext] N$ if and only if $M \equal[\eta] N$.
\end{thm}

\begin{proof}
  First we prove that $\equal[\eta]$ is closed under the
  extensionality rule. That is, $ext$ rule doesn't add anything to
  $\equal[\eta]$. We then have $\equal[\eta]$ contains $\equal[\ext]$,
  and if $M \equal[\ext] N$, then $M \equal[\eta] N$.

  To prove $\equal[\eta]$ is closed under \ext, note that for any $M
  \equal N$ derived by the \ext{} rule, we have $Mx
  \equal[\eta] Nx$ as premise. Then we have $\lambd[x][Mx]
  \equal[\eta] \lambd[x][Nx]$ by a rule of $\equal$, applying $\eta$
  on both side gives us $M \equal[\eta] N$.

  Similarly we prove that the $\eta$ rule is contained in
  $\equal[ext]$. For any $\lambd[x][Mx]$ and $M$ with $x \notin
  FV(M)$, we have that $(\lambd[x][Mx])x \equal[\ext] Mx$,
  giving us $\lambd[x][Mx] \equal[\ext] M$ by the \ext{} rule.
\end{proof}

\end{document}
