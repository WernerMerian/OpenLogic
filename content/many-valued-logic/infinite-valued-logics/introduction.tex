% Part: many-valued-logic
% Chapter: infinite-valued-logics
% Section: introduction

\documentclass[../../../include/open-logic-section]{subfiles}

\begin{document}

\olsection{mvl}{inf}{int}{Introduction}

The number of truth values of a matrix need not be finite.  An obvious
choice for a set of infinitely many truth values is the set of
rational numbers between $0$ and~$1$, $V_\infty = [0,1] \cap \Rat$,
i.e.,
\begin{align*}
    V_\infty & = \Setabs{\frac{n}{m}}{n,m \in \Nat \text{ and } n\le m}.
\intertext{When considering this infinite truth value set, it is often
useful to also consider the subsets}
V_m & = \Setabs{\frac{n}{m-1}}{n \in \Nat \text{ and } n\le m}
\intertext{For instance, $V_5$ is the set with $5$ evenly spaced truth values,}
V_5 & = \{0, \frac{1}{4}, \frac{1}{2}, \frac{3}{4}, 1\}.
\end{align*}
In logics based on these truth value sets, usually only $1$ is
designated, i.e., $V^+ = \{1\}$.  In other words, we let $1$ play the
role of (absolute) truth, $0$ as absolute falsity, but !!{formula}s
may take any intermediate value in~$V$.

One can also consider the set $V_{[0,1]} = [0,1]$ of all
\emph{real} numbers between $0$ and~$1$, or other infinite subsets of
$[0,1]$, however. Logics with this truth value set are often called \emph{fuzzy}.


\end{document}
