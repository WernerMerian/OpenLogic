% Part: many-valued-logic
% Chapter: three-valued-logics
% Section: multiple-designation

\documentclass[../../../include/open-logic-section]{subfiles}

\begin{document}

\olsection{mvl}{thr}{mul}{Designating not just $\True$}

So far the logics we've seen all had the set of designated truth
values $V^+ = \{\True\}$, i.e., something counts as true iff its truth
value is~$\True$.  But one might also count something as true if it's
just not~$\False$. Then one would get a logic by stipulating in the
matrix, e.g., that $V^+ = \{\True, \Undef\}$.

\begin{defn}
The \emph{logic of paradox}~$\LogLP$ is defined using the matrix:
\begin{enumerate}
  \item The standard propositional language $\Lang L_0$ with
  $\lnot$, $\land$, $\lor$, $\lif$.
  \item The set of truth values $V = \{\True, \Undef, \False\}$.
  \item $\True$ and $\Undef$ are designated, i.e., $V^+ = \{\True, \Undef\}$.
  \item Truth functions are the same as in strong Kleene logic.
\end{enumerate}
\end{defn}

\begin{defn}
Halld\'en's \emph{logic of nonsense}~$\LogHal$ is defined using the matrix:
\begin{enumerate}
  \item The standard propositional language $\Lang L_0$ with
  $\lnot$, $\land$, $\lor$, $\lif$ and a $1$-place connective~$+$.
  \item The set of truth values $V = \{\True, \Undef, \False\}$.
  \item $\True$ and $\Undef$ are designated, i.e., $V^+ = \{\True, \Undef\}$.
  \item Truth functions are the same as weak Kleene logic, plus the
  ``is meaningless'' operator:
  \begin{center}
    \begin{tabular}{c|c} 
    $\tf{+}$ & \\ 
    \hline  
    $\True$ & $\False$ \\ 
    $\Undef$ & $\True$ \\
    $\False$ & $\False$ 
    \end{tabular}
  \end{center}
\end{enumerate}
\end{defn}

By contrast to the Kleene logics with which they share truth tables,
these \emph{do} have tautologies. 

\begin{prop}\ollabel{prop:LP-taut-CL} The tautologies of $\LogLP$  are
  the same as the tautologies of classical propositional logic.
\end{prop}

\begin{proof}
  By \olref[syn][sub]{prop:mvl-cl}, if $\Entails[\LogLP] !A$ then
  $\Entails[\LogCL] !A$. To show the reverse, we show that if there is
  !!a{valuation}~$\pAssign v\colon \PVar \to \{\False, \True,
  \Undef\}$ such that $\pValue v(!A)[\LogKs] = \False$ then there is
  !!a{valuation}~$\pAssign {v'}\colon \PVar \to \{\False, \True\}$
  such that $\pValue {v'}(!A)[\LogCL] = \False$. This establishes the
  result for $\LogLP$, since $\LogKs$ and $\LogLP$ have the same
  characteristic truth functions, and $\False$ is the only truth value
  of $\LogLP$ that is not designated (that is the only difference
  between $\LogLP$ and~$\LogKs$). Thus, if $\Entails/[\LogLP] !A$, for
  some !!{valuation}~$\pAssign v$, $\pValue v(!A)[\LogLP] = \pValue
  v[\LogKs](!A) = \False$. By the claim we're proving,
  $\pValue{v'}[\LogCL](!A) = \False$, i.e., $\Entails/[\LogCL] !A$.
  
  To establish the claim, we first define $\pAssign {v'}$ as
  \[
    \pAssign {v'}(p) = 
    \begin{cases}
    \True & \text{if } \pAssign {v}(p) \in \{\True, \Undef\}\\
    \False & \text{otherwise}
  \end{cases}
  \]
  We now show by induction on $!A$ that (a)~if $\pValue v(!A)[\LogKs]
  = \False$ then $\pValue {v'}(!A)[\LogCL] = \False$, and (b)~if
  $\pValue v(!A)[\LogKs] = \True$ then $\pValue {v'}(!A)[\LogCL] =
  \True$
  \begin{enumerate}
    \item Induction basis: $!A \ident p$. By
    \olref[syn][val]{defn:pValue}, $\pValue v(!A)[\LogKs]
    = \pAssign v(p) = \pValue {v'}(!A)[\LogCL]$, which implies both (a)
    and~(b).
    
    For the induction step, consider the cases:
    \item $!A \ident \lnot !B$. 
    \begin{enumerate}
      \item Suppose $\pValue v(\lnot!B)[\LogKs] = \False$. By the
      definition of $\tf{\lnot}[\LogKs]$, $\pValue v(!B)[\LogKs]  =
      \True$. By inductive hypothesis, case~(b), we get $\pValue
      {v'}(!B)[\LogCL]  = \True$, so $\pValue {v'}(\lnot !B)[\LogCL] = \False$.
      \item Suppose $\pValue v(\lnot!B)[\LogKs] = \True$. By the
      definition of $\tf{\lnot}[\LogKs]$, $\pValue v(!B)[\LogKs]  =
      \False$. By inductive hypothesis, case~(a), we get $\pValue
      {v'}(!B)[\LogCL]  = \False$, so $\pValue {v'}(\lnot !B)[\LogCL] = \True$.
    \end{enumerate}

    \item $!A \ident (!B \land !C)$.
    \begin{enumerate}
      \item Suppose $\pValue v(!B \land !C)[\LogKs] = \False$. By the
      definition of $\tf{\land}[\LogKs]$, $\pValue v(!B)[\LogKs]  =
      \False$ or $\pValue v(!B)[\LogKs]  = \False$. By inductive
      hypothesis, case~(a), we get $\pValue {v'}(!B)[\LogCL]  = \False$
      or $\pValue {v'}(!C)[\LogCL]  = \False$, so $\pValue {v'}(!B \land
      !C)[\LogCL] = \False$.
      \item Suppose $\pValue v(!B \land !C)[\LogKs] = \True$. By the
      definition of $\tf{\land}[\LogKs]$, $\pValue v(!B)[\LogKs]  =
      \True$ and $\pValue v(!B)[\LogKs]  = \True$. By inductive
      hypothesis, case~(b), we get $\pValue {v'}(!B)[\LogCL]  = \True$
      and $\pValue {v'}(!C)[\LogCL]  = \True$, so $\pValue {v'}(!B \land
      !C)[\LogCL] = \True$.
    \end{enumerate}
  \end{enumerate}

  The other two cases are similar, and left as exercises. Alternatively,
  the proof above establishes the result for all !!{formula}s only
  containing $\lnot$ and~$\land$. One may now appeal to the facts that in
  both $\LogKs$ and $\LogCL$, for any $\pAssign v$, $\pValue v(!B \lor
  !C) = \pValue v(\lnot(\lnot!B \land \lnot !C))$ and $\pValue v(!B \lif
  !C) = \pValue v(\lnot(!B \land \lnot !C))$.
\end{proof}

\begin{prob}
  Complete the proof \olref[mvl][thr][mul]{prop:LP-taut-CL}, i.e.,
  establish (a) and~(b) for the cases where $!A \ident (!B \lor !C)$
  and $!A \ident (!B \lif !C)$.
\end{prob}

\begin{prob}
Prove that every classical tautology is a tautology in~$\LogHal$.
\end{prob}

Although they have the same tautologies as classical logic, their
consequence relations are different.  $\LogLP$, for instance, is
\emph{paraconsistent} in that $\lnot p, p \Entails/ q$, and so the
principle of explosion $\lnot !A, !A \Entails !B$ does not hold in
general. (It holds for some cases of $!A$ and $!B$, e.g., if $!B$~is a
tautology.)

\begin{prob}
  Which of the following relations hold in (a)~$\LogLP$
  and in (b)~$\LogHal$? Give a truth table for each.
  \begin{enumerate}
    \item $p, p \lif q \Entails q$
    \item $\lnot q, p \lif q \Entails \lnot p$
    \item $p \lor q, \lnot p \Entails q$
    \item $\lnot p, p \Entails q$
    \item $p \Entails p \lor q$
    \item $p \lif q, q\lif r \Entails p \lif r$
  \end{enumerate}
\end{prob}

What if you make $\Undef$ designated in $\LogLuk[3]$?

\begin{defn}
  The logic \emph{3-valued R-Mingle}~$\LogRM[3]$ is defined using the matrix:
  \begin{enumerate}
    \item The standard propositional language $\Lang L_0$ with
    $\lfalse$, $\lnot$, $\land$, $\lor$, $\lif$.
    \item The set of truth values $V = \{\True, \Undef, \False\}$.
    \item $\True$ and $\Undef$ are designated, i.e., $V^+ = \{\True, \Undef\}$.
    \item Truth functions are the same as \L ukasiewicz logic~$\LogLuk[3]$.
  \end{enumerate}
\end{defn}

\begin{prob}
  Which of the following relations hold in $\LogRM[3]$?
  \begin{enumerate}
    \item $p, p \lif q \Entails q$
    \item $p \lor q, \lnot p \Entails q$
    \item $\lnot p, p \Entails q$
    \item $p \Entails p \lor q$
  \end{enumerate}
\end{prob}

Different truth tables can sometimes generate the same logic
(entailment relation) just by changing the designated values.
E.g., this happens if in G\"odel logic we take $V^+ = \{\True,
\Undef\}$ instead of~$\{\True\}$.

\begin{prop}\ollabel{prop:gl-udes}
  The matrix with $V = \{\False, \Undef,\True\}$, $V^+=\{\True,
  \Undef\}$, and the truth functions of $3$-valued G\"odel logic
  defines classical logic.
\end{prop}

\begin{proof}
  Exercise. 
\end{proof}

\begin{prob}
  Prove \olref[mvl][thr][mul]{prop:gl-udes} by showing that for the
  logic~$\Log L$ defined just like G\"odel logic but with
  $V^+=\{\True,\Undef\}$, if $\Gamma \Entails/[\Log L] !B$ then
  $\Gamma \Entails/[\LogCL] !B$. Use the ideas of
  \olref[mvl][thr][mul]{prop:LP-taut-CL}, except instead of proving
  properties (a) and~(b), show that $\pValue v(!A)[\LogGod] = \False$
  iff $\pValue {v'}(!A)[\LogCL] = \False$ (and hence that $\pValue
  v(!A)[\LogGod] \in \{\True,\Undef\}$ iff $\pValue {v'}(!A)[\LogCL] =
  \True$). Explain why this establishes the proposition.
\end{prob}

\end{document}
