% Part: methods
% Chapter: proofs
% Section: starting-proofs

\documentclass[../../../include/open-logic-section]{subfiles}

\begin{document}

\olsection{mth}{prf}{str}{Starting a Proof}

But where do you even start?

You've been given something to prove, so this should be the last thing that
is mentioned in the proof (you can, obviously, \emph{announce} that you're
going to prove it at the beginning, but you don't want to use it as an
assumption). Write what you are trying to prove at the bottom of a fresh
sheet of paper---this way you don't lose sight of your goal.

Next, you may have some assumptions that you are able to use (this
will be made clearer when we talk about the \emph{type} of proof you
are doing in the next section). Write these at the top of the page and
make sure to flag that they are assumptions (i.e., if you are assuming
$p$, write ``assume that $p$,'' or ``suppose that $p$''). Finally,
there might be some definitions in the question that you need to know.
You might be told to use a specific definition, or there might be
various definitions in the assumptions or conclusion that you are
working towards. \emph{Write these down and ensure that you understand
what they mean.}

How you set up your proof will also be dependent upon the form of the
question. The next section provides details on how to set up your proof
based on the type of sentence.

\end{document}
