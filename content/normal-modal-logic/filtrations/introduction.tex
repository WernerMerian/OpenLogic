% Part: normal-modal-logic
% Chapter: filtrations
% Section: introduction

\documentclass[../../../include/open-logic-section]{subfiles}

\begin{document}

\olsection{nml}{fil}{int}{Introduction}

One important question about a logic is always whether it is
decidable, i.e., if there is an effective procedure which will answer
the question ``is this !!{formula} valid.'' Propositional logic is
decidable: we can effectively test if !!a{formula} is a tautology by
constructing a truth table, and for a given !!{formula}, the truth
table is finite.  But we can't obviously test if a modal formula is
true in all models, for there are infinitely many of them.  We can
list all the finite models relevant to a given formula, since only the
assignment of subsets of worlds to !!{propositional variable}s which
actually occur in the !!{formula} are relevant. If the accessibility
relation is fixed, the possible different assignments $V(p)$ are just
all the subsets of~$W$, and if $\card{W} = n$ there are $2^n$ of
those. If our !!{formula}~$!A$ contains $m$ !!{propositional
  variable}s there are then $2^{nm}$ different models with~$n$
worlds. For each one, we can test if~$!A$ is true at all worlds,
simply by computing the truth value of~$!A$ in each. Of course, we
also have to check all possible accessibility relations, but there are
only finitely many relations on~$n$ worlds as well (specifically, the
number of subsets of $W \times W$, i.e., $2^{n^2}$.

If we are not interested in the logic~\Log{K}, but a logic defined by
some class of models (e.g., the reflexive transitive models), we also
have to be able to test if the accessibility relation is of the right
kind. We can do that whenever the frames we are interested in are
definable by modal !!{formula}s (e.g., by testing if \Ax{T} and \Ax{4}
valid in the frame). So, the idea would be to run through all the
finite frames, test each one if it is a frame in the class we're
interested in, then list all the possible models on that frame and
test if $!A$ is true in each. If not, stop: $!A$ is not valid in the
class of models of interest.

There is a problem with this idea: we don't know when, if ever, we can
stop looking. If the !!{formula} has a finite countermodel, our procedure will
find it. But if it has no finite countermodel, we won't get an
answer. The !!{formula} may be valid (no countermodels at all), or it
have only an infinite countermodel, which we'll never look at. This
problem can be overcome if we can show that every !!{formula} that has
a countermodel has a finite countermodel. If this is the case we say
the logic has the \emph{finite model property}.

But how would we show that a logic has the finite model property? One
way of doing this would be to find a way to turn an infinite
(counter)model of~$!A$ into a finite one. If that can be done, then
whenever there is a model in which $!A$ is not true, then the
resulting finite model also makes $!A$ not true. That finite model
will show up on our list of all finite models, and we will eventually
determine, for every formula that is not valid, that it isn't. Our
procedure won't terminate if the formula is valid. If we can show in
addition that there is some maximum size that the finite model our
procedure provides can have, and that this maximum size depends only
on the !!{formula}~$!A$, we will have a size up to which we have to
test finite models in our search for countermodels. If we haven't
found a countermodel by then, there are none. Then our procedure will,
in fact, decide the question ``is $!A$ valid?'' for any
!!{formula}~$!A$.

A strategy that often works for turning infinite structures into
finite structures is that of ``identifying'' !!{element}s of the
structure which behave the same way in relevant respects. If there are
infinitely many worlds in~$\mModel{M}$ that behave the same in
relevant respects, then we might hope that there are only finitely
many ``classes'' of such worlds. In other words, we partition the set
of worlds in the right way. Each partition contains infinitely many
worlds, but there are only finitely many partitions. Then we define a
new model~$\mModel{M^*}$ where the worlds are the partitions. Finitely
many partitions in the old model give us finitely many worlds in the
new model, i.e., a finite model. Let's call the partition a world~$w$
is in $[w]$. We'll want it to be the case that $\mSat{M}{!A}[w]$ iff
$\mSat{M^*}{!A}[{[w]}]$, since we want the new model to be a
countermodel to~$!A$ if the old one was. This requires that we define
the partition, as well as the accessibility relation of~$\mModel{M^*}$
in the right way.

To see how this would go, first imagine we have no accessibility
relation. $\mSat{M}{\Box !B}[w]$ iff for some $v \in W$,
$\mSat{M}{\Box !B}[v]$, and the same for $\mModel{M^*}$, except with
$[w]$ and $[v]$. As a first idea, let's say that two worlds $u$ and
$v$ are equivalent (belong to the same partition) if they agree on all
!!{propositional variable}s in~$\mModel{M}$, i.e., $\mSat{M}{p}[u]$
iff $\mSat{M}{p}[v]$. Let $V^*(p) = \Setabs{[w]}{\mSat{M}{p}[w]}$.
Our aim is to show that $\mSat{M}{!A}[w]$ iff
$\mSat{M^*}{!A}[{[w]}]$. Obviously, we'd prove this by induction: The
base case would be $!A \equiv p$. First suppose $\mSat{M}{p}[w]$. Then
$[w] \in V^*$ by definition, so $\mSat{M^*}{p}[{[w]}]$. Now suppose
that $\mSat{M^*}{p}[{[w]}]$. That means that $[w] \in V^*(p)$, i.e.,
for some $v$ equivalent to~$w$, $\mSat{M}{p}[v]$. But ``$w$ equivalent
to $v$'' means ``$w$ and $v$ make all the same !!{propositional variable}s
true,'' so $\mSat{M}{p}[w]$. Now for the inductive step, e.g., $!A
\ident \lnot !B$. Then $\mSat{M}{\lnot !B}[w]$ iff $\mSat/{M}{!B}[w]$
iff $\mSat/{M^*}{!B}[{[w]}]$ (by inductive hypothesis) iff
$\mSat{M^*}{\lnot !B}[{[w]}]$. Similarly for the other non-modal
operators. It also works for $\Box$: suppose $\mSat{M^*}{\Box
  !B}[{[w]}]$. That means that for every $[u]$,
$\mSat{M^*}{!B}[{[u]}]$. By inductive hypothesis, for every $u$,
$\mSat{M}{!B}[u]$. Consequently, $\mSat{M}{\Box !B}[w]$.

In the general case, where we have to also define the accessibility
relation for $\mModel{M^*}$, things are more complicated. We'll call a
model~$\mModel{M^*}$ a \emph{filtration} if its accessibility
relation~$R^*$ satisfies the conditions required to make the inductive
proof above go through. Then any filtration~$\mModel{M^*}$ will make
$!A$ true at $[w]$ iff $\mModel{M}$ makes $!A$ true at~$w$. However,
now we also have to show that there \emph{are} filtrations, i.e., we
can define~$R^*$ so that it satisfies the required conditions. In
order for this to work, however, we have to require that worlds $u$,
$v$ count as equivalent not just when they agree on all
!!{propositional variable}s, but on all sub-!!{formula}s
of~$!A$. Since $!A$ has only finitely many sub-!!{formula}s, this will
still guarantee that the filtration is finite. There is not just one
way to define a filtration, and in order to make sure that the
accessibility relation of the filtration satisfies the required
properties (e.g., reflexive, transitive, etc.) we have to be inventive
with the definition of~$R^*$.


\end{document}


