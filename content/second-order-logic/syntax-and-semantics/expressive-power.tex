% Part: second-order-logic
% Chapter: syntax-and-semantics
% Section: expressive-power

\documentclass[../../../include/open-logic-section]{subfiles}

\begin{document}

\olsection{sol}{syn}{exp}{Expressive Power}

\begin{explain}
Quantification over second-order variables is responsible for an
immense increase in the expressive power of the language over that of
first-order logic.  Second-order existential quantification lets us
say that functions or relations with certain properties exists. In
first-order logic, the only way to do that is to specify a non-logical
symbol (i.e., !!a{function} or !!{predicate}) for this
purpose. Second-order universal quantification lets us say that all
subsets of, relations on, or functions from the !!{domain} to the
!!{domain} have a property.  In first-order logic, we can only say
that the subsets, relations, or functions assigned to one of the
non-logical symbols of the language have a property.  And when we say
that subsets, relations, functions exist that have a property, or that
all of them have it, we can use second-order quantification in
specifying this property as well. This lets us define relations not
definable in first-order logic, and express properties of the domain
not expressible in first-order logic.
\end{explain}

\begin{defn}
If $\Struct{M}$ is !!a{structure} for a language~$\Lang{L}$, a
relation~$R \subseteq \Domain{M}^2$ is \emph{definable} in~$\Lang{L}$
if there is some !!{formula}~$!A_R(x, y)$ with only the variables $x$
and~$y$ free, such that $R(a, b)$ holds (i.e., $\tuple{a,b} \in R$)
iff $\Sat{M}{!A_R(x, y)}[s]$ for $s(x) = a$ and $s(y) = b$.
\end{defn}

\begin{ex}
In first-order logic we can define the identity
relation~$\Id{\Domain{M}}$ (i.e., $\Setabs{\tuple{a,a}}{a \in
  \Domain{M}}$) by the formula $\eq[x][y]$.  In
second-order logic, we can define this relation
\emph{without~$\eq$}. For if $a$ and $b$ are the same !!{element}
of~$\Domain{M}$, then they are !!{element}s of the same subsets
of~$\Domain{M}$ (since sets are determined by their
!!{element}s). Conversely, if $a$ and $b$ are different, then they are
not !!{element}s of the same subsets: e.g., $a \in \{a\}$ but $b
\notin \{a\}$ if $a \neq b$.  So ``being !!{element}s of the same
subsets of~$\Domain{M}$'' is a relation that holds of $a$ and $b$ iff
$a = b$. It is a relation that can be expressed in second-order logic,
since we can quantify over all subsets of~$\Domain{M}$. Hence, the
following !!{formula} defines $\Id{\Domain{M}}$:
\[
\lforall[X][(X(x) \liff X(y))]
\]
\end{ex}

\begin{prob}
Show that $\lforall[X][(X(x) \lif X(y))]$ (note:
$\lif$ not~$\liff$!) defines $\Id{\Domain{M}}$.
\end{prob}

\begin{ex}
If $R$ is a two-place !!{predicate}, $\Assign{R}{M}$ is a two-place
relation on~$\Domain{M}$.  Perhaps somewhat confusingly, we'll use $R$
as the !!{predicate} for~$R$ and for the relation~$\Assign{R}{M}$
itself.  The \emph{transitive closure}~$R^*$ of~$R$ is the relation
that holds between $a$ and $b$ iff for some $c_1$, \dots, $c_k$,
$R(a,c_1)$, $R(c_1, c_2)$, \dots, $R(c_k,b)$ holds. This includes the
case if $k = 0$, i.e., if $R(a,b)$ holds, so does $R^*(a,b)$. This
means that $R \subseteq R^*$. In fact, $R^*$ is the smallest relation
that includes~$R$ and that is transitive.  We can say in second-order
logic that $X$ is a transitive relation that includes~$R$:
\begin{multline*}
  !B_R(X) \ident \lforall[x][\lforall[y][(R(x,y) \lif X(x, y))]] \land {}\\
\lforall[x][\lforall[y][\lforall[z][((X(x,y) \land X(y,z)) \lif X(x,
      z))]]].
\end{multline*}
The first conjunct says that $R \subseteq X$ and the second
that $X$ is transitive.

To say that $X$ is the smallest such relation is to say that it is
itself included in every relation that includes $R$ and is
transitive. So we can define the transitive closure of~$R$ by the
!!{formula}
\[
R^*(X) \ident !B_R(X) \land \lforall[Y][(!B_R(Y) \lif
  \lforall[x][\lforall[y][(X(x, y) \lif Y(x,y))]])].
\]
We have $\Sat{M}{R^*(X)}[s]$ iff $s(X) = R^*$. The transitive closure
of~$R$ cannot be expressed in first-order logic.
\end{ex}

\end{document}
