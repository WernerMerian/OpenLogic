\documentclass[../../../include/open-logic-section]{subfiles}

\begin{document}

\olsection{sth}{cardinals}{hp}{Appendix: Hume's Principle}

In \olref[cp]{sec}, we described Cantor's Principle. This was:
\begin{align*}
	\card{A} = \card{B} & \text{ iff } A \approx B.
\intertext{This is very similar to what is now called
\emph{Hume's Principle}, which says:}
	\fregenum{x} {F(x)} = \fregenum{x}{G(x)} & \text{ iff } F \sim G
\end{align*}
where `$F \sim G$' abbreviates that there are exactly as many $F$s as
$G$s, i.e., the $F$s can be put into a bijection with the $G$s, i.e.:
\begin{align*}
	\exists R(&\forall v\forall y(Rvy \lif (Fv \land Gy)) \land {}\\
		&\forall v(Fv \lif \lexists![y][Rvy]) \land {}\\
		&\forall y(Gy \lif \lexists![v][Rvy]))
\end{align*}
But there is a type-difference between Hume's Principle and Cantor's
Principle. In the statement of Cantor's Principle, the variables
``$A$'' and ``$B$'' are first-order terms which stand for \emph{sets}.
In the statement of Hume's Principle, ``$F$'', ``$G$'' and ``$R$'' are
\emph{not} first-order terms; rather, they are in \emph{predicate
position}. (Maybe they stand for \emph{properties}.) So we might gloss
Hume's Principle in English as: the number of $F$s is the number of
$G$s iff the $F$s are bijective with the~$G$s. This is called
\emph{Hume's Principle}, because Hume once wrote this:
\begin{quote}
  When two numbers are so combined as that the one has always an unit
  answering to every unit of the other, we pronounce them equal.
  \cite[Pt.III Bk.1 \S1]{Hume1740}
\end{quote}
And Hume's Principle was brought to contemporary mathematico-logical
prominence by \cite[\S63]{Frege1884}, who quoted this passage from
Hume, before (in effect) sketching (what we have called) Hume's
Principle. 

You should note the structural similarity between Hume's Principle and
Basic Law~V. We formulated this in \olref[story][blv]{sec} as
follows:
\[
	\fregeext{x}{F(x)} = \fregeext{x}{G(x)} \text{iff } \lforall[x][(F(x) \liff G(x))].
\]
And, at this point, some commentary and comparison might help. 

There are two ways to take a principle like Hume's Principle or Basic
Law~V: \emph{predicatively} or \emph{impredicatively} (recall
\olref[story][predicative]{sec}). On the impredicative reading of
Basic Law~V, for each~$F$, the object $\fregeext{x}{F(x)}$ falls
within the domain of quantification that we used in formulating Basic
Law~V itself. Similarly, on the impredicative reading of Hume's
Principle, for each~$F$, the object $\fregenum{x}{F(x)}$ falls within
the domain of quantification that we used in formulating Hume's
Principle. By contrast, on the \emph{predicative} understanding, the
objects $\fregeext{x}{F(x)}$ and~$\fregenum{x}{F(x)}$ would be
entities from some \emph{different} domain. 

Now, if we read Basic Law~V impredicatively, it leads to
inconsistency, via Na\"ive Comprehension (for the details, see
\olref[story][blv]{sec}). Much like Na\"ive Comprehension, it can be
rendered consistent by reading it \emph{predicatively}. But it
probably will not do everything that we wanted it to. 

Hume's Principle, however, \emph{can} consistently be read
impredicatively. And, read thus, it is quite powerful.

To illustrate: consider the predicate ``$x \neq x$'', which obviously
nothing satisfies. Hume's Principle now yields an object $\# x( x\neq
x)$. We might treat this as the number~$0$. Now, on the
\emph{impredicative} understanding---but \emph{only} on the
impredicative understanding---this entity $0$ falls within our
original domain of quantification. So we can sensibly apply Hume's
Principle with the predicate ``$x = 0$'' to obtain an object $\#x (x =
0)$. We might treat this as the number~$1$. Moreover, Hume's Principle
entails that $0 \neq 1$, since there cannot be a bijection from the
non-self-identical objects to the objects identical with $0$ (there
are none of the former, but one of the latter). Now, working
impredicatively again, $1$~falls within our original domain of
quantification. So we can sensibly apply Hume's Principle with the
predicate ``$(x = 0 \lor x = 1)$'' to obtain an object $\#x(x = 0 \lor
x = 1)$. We might treat this as the number~$2$, and we can show that
$0\neq 2$ and $1 \neq 2$ and so on. 

In short, taken impredicatively, Hume's Principle entails that there
are \emph{infinitely many objects}. And this has encouraged
\emph{neo-Fregean logicists} to take Hume's Principle as the
foundation for arithmetic. 

Frege \emph{himself}, though, did not take Hume's Principle as his
foundation for arithmetic. Instead, Frege proved Hume's Principle from
an explicit definition: $\fregenum{x}{F(x)}$ is defined as the extension of
the concept $F \sim \Phi$. In modern terms, we might attempt to render
this as $\fregenum{x}{F(x)} = \Setabs{G}{F \sim G}$; but this will pull us
back into the problems of Na\"ive Comprehension.

\end{document}