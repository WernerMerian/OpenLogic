\documentclass[../../../include/open-logic-section]{subfiles}

\begin{document}

\olsection{sth}{choice}{countablechoice}{Countable Choice}

It is easy to prove, without any use of Choice/Well-Ordering, that:

\begin{lem}[in $\Zminus$]
Every finite set has a choice function. 
\end{lem}

\begin{proof}
Let $a = \{b_1, \ldots, b_n\}$. Suppose for simplicity that each $b_i
\neq \emptyset$. So there are objects $c_1, \ldots, c_n$ such that
$c_1 \in b_1, \ldots, c_n \in b_n$. Now by
\olref[z][pairs]{prop:pairsconsequences}, the set $\{\langle b_1,
c_1\rangle , \ldots, \langle b_n, c_n\rangle\}$ exists; and this is a
choice function for~$a$.
\end{proof}

But matters get murkier as soon as we consider infinite sets. For
example, consider this ``minimal'' extension to the above:

\begin{defish}
\emph{Countable Choice.} Every \emph{countable} set has a choice function. 
\end{defish}

This is a special case of Choice. And it transpires that this
principle was invoked fairly frequently, without  an obvious awareness
of its use. Here are two nice examples.\footnote{Due to
\cite[\S9.4]{Potter2004} and Luca Incurvati.}

\begin{ex}
Here is a natural thought: for any set $A$, either
$\cardle{\omega}{A}$, or $\cardeq{A}{n}$ for some $n \in \omega$. This
is one way to state the intuitive idea, that every set is either
finite or infinite. Cantor, and many other mathematicians, made this
claim without proving it. Cautious as we are, we proved this in
\olref[cardinals][classing]{generalinfinitycharacter}. But
in that proof we were working in $\ZFC$, since we were assuming that
any set $A$ can be well-ordered, and hence that $\card{A}$ is
guaranteed to exist. That is: we explicitly assumed Choice.

In fact, \cite{Dedekind1888} offered his own proof of
this claim, as follows:

\begin{thm}[in $\Zminus + \text{Countable Choice}$]
For any $A$, either $\cardle{\omega}{A}$ or $\cardeq{A}{n}$ for some
$n \in \omega$.
\end{thm}

\begin{proof}
Suppose $\cardneq{A}{n}$ for all $n \in \omega$. Then in particular
for each $n < \omega$ there is subset $A_n \subseteq A$ with exactly
$\cardexpo{2}{n}$ elements. Using this sequence $A_0, A_1, A_2,
\ldots$, we define for each $n$:
\[
	B_n = A_n \setminus \bigcup_{i < n} A_i.
\]
Now note the following
\begin{align*}
	\card{\bigcup_{i < n}A_n} 
	&\leq \card{A_0} + \card{A_1} + \ldots + \card{A_{n-1}}\\
	&=1 + 2 + \ldots + 2^{n-1}\\
	& = 2^n - 1\\
	& < 2^n = \card{A_n}
\end{align*}
Hence each $B_n$ has at least one member, $c_n$. Moreover, the $B_n$s
are pairwise disjoint; so if $c_n = c_m$ then $n = m$. But every $c_n
\in A$. So the function  $f(n) = c_n$ is an injection $\omega \to A$.
\end{proof}
\noindent 
Dedekind did not flag that he had used Countable Choice. But, did
\emph{you} spot its use? Look again. (Really: \emph{look again}.)

The proof used Countable Choice twice. We used it once, to obtain
our sequence of sets $A_0$, $A_1$, $A_2$, \dots\@ We then used it
again to select our elements $c_n$ from each~$B_n$. Moreover, this use
of Choice is ineliminable. \cite[p.~138]{Cohen1966} proved that the
result fails if we have no version of Choice. That is: it is
consistent with $\ZF$ that there are sets which are
\emph{incomparable} with~$\omega$.
\end{ex}

\begin{ex} 
In \cite{Cantor1878}, Cantor stated that a countable union of
countable sets is countable. He did not present a proof, perhaps
indicating that he took the proof to be obvious. Now, cautious as we
are, we proved a more general version of this result in
\olref[card-arithmetic][simp]{kappaunionkappasize}.  But our proof
explicitly assumed Choice. And even the proof of the less general
result requires Countable Choice.

\begin{thm}[in $\Zminus + \text{Countable Choice}$]
If $A_n$ is countable for each $n \in \omega$, then $\bigcup_{n <
\omega} A_n$ is countable.
\end{thm}

\begin{proof}
Without loss of generality, suppose that each $A_n \neq \emptyset$. So
for each $n \in \omega$ there is !!a{surjection} $f_n \colon \omega
\to A_n$. Define $f \colon \omega \times \omega \to \bigcup_{n <
\omega} A_n$ by $f(m, n) = f_n(m)$. The result follows because $\omega
\times \omega$ is countable
(\olref[sfr][siz][zigzag]{natsquaredenumerable}) and $f$ is
!!a{surjection}.
\end{proof}
\noindent 
Did you spot the use of the Countable Choice? It is used to choose our
sequence of functions $f_0$, $f_1$, $f_2$, \dots\footnote{A similar
use of Choice occurred in
\olref[card-arithmetic][simp]{kappaunionkappasize}, when we gave the
instruction ``For each $\beta \in \cardfont{a}$, fix !!a{injection}
$f_\beta$''.} And again, the result fails in the absence of any Choice
principle. Specifically, \cite{FefermanLevy1963} proved that it is
consistent with $\ZF$ that a countable union of countable sets has
cardinality~$\beth_1$. But here is a much funnier statement of the
point, from Russell:
\begin{quote}
  This is illustrated by the millionaire who bought a pair of socks
  whenever he bought a pair of boots, and never at any other time, and
  who had such a passion for buying both that at last he had
  $\aleph_0$ pairs of boots and $\aleph_0$ pairs of socks\dots\@ Among
  boots we can distinguish right and left, and therefore we can make a
  selection of one out of each pair, namely, we can choose all the
  right boots or all the left boots; but with socks no such principle
  of selection suggests itself, and we cannot be sure, unless we
  assume the multiplicative axiom [i.e., in effect Choice], that there
  is any class consisting of one sock out of each pair.
  \cite[p.~126]{Russell1919}
\end{quote}
In short, some form of Choice is needed to prove the following: If you
have countably many pairs of socks, then you have (only) countably
many socks. And in fact, without Countable Choice (or something
equivalent), a countable union of countable sets can fail to be
countable. 
\end{ex}

The moral is that Countable Choice was used repeatedly, without much
awareness of its users. The philosophical question is: How could we
\emph{justify} Countable Choice? 

An attempt at an intuitive justification might invoke an appeal to a
supertask. Suppose we make the first choice in $\nicefrac{1}{2}$ a
minute, our second choice in $\nicefrac{1}{4}$ a minute, \dots, our
$n$-th choice in $\nicefrac{1}{2^n}$ a minute, \dots\@ Then within $1$~minute, we will have made an $\omega$-sequence of choices, and defined
a choice function. 

But what, really, could such a thought-experiment tell us? For a
start, it relies upon taking this idea of ``choosing'' rather
literally. For another, it seems to bind up mathematics in
metaphysical possibility. 

More important: it is not going to give us any justification for
Choice \emph{tout court}, rather than \emph{mere} Countable Choice.
For if we need \emph{every} set to have a choice function, then we'll
need to be able to perform a ``supertask of arbitrary ordinal
length.'' Bluntly, that idea is laughable.

\end{document}