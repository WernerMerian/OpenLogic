\documentclass[../../../include/open-logic-section]{subfiles}

\begin{document}

\olsection{sth}{ordinals}{ordtype}{Ordinals as Order-Types}

Armed with Replacement, and so now working in $\ZFminus$, we can finally prove the result we have been aiming for:

\begin{thm}\ollabel{thmOrdinalRepresentation}
Every well-ordering is isomorphic to a unique ordinal. 
\end{thm}

\begin{proof}
Let $\tuple{A, <}$ be a well-order. By
\olref[basic]{ordisoidentity}, it is isomorphic to at most one
ordinal. So, for reductio, suppose $\tuple{A, <}$ is not isomorphic to
\emph{any} ordinal. We will first ``make $\tuple{A, <}$ as small as
possible''. In detail: if some proper initial segment $\tuple{A_a,
<_a}$ is not isomorphic to any ordinal, there is a least $a \in A$
with that property; then let $B = A_a$ and $\mathord{\lessdot} =
\mathord{<_a}$. Otherwise, let $B = A$ and $\mathord{\lessdot} =
\mathord{<}$. 

By definition, every proper initial segment of $B$ is isomorphic to
some ordinal, which is unique as above. So
by Replacement, the following set exists, and is a function:
\[
	f = \Setabs{\tuple{\beta, b}}{b \in B\text{ and }\ordeq{\beta}{\tuple{B_b, \lessdot_b}}}
\]
To complete the reductio, we'll show that $f$ is an isomorphism
$\alpha \to B$, for some ordinal $\alpha$. 

It is obvious that $\ran{f}
= B$. And by \olref[iso]{lemordsegments}, $f$ preserves ordering,
i.e., $\gamma \in \beta$ iff $f(\gamma) \lessdot f(\beta)$. To show that $\dom{f}$ is an ordinal, by
\olref[basic]{corordtransitiveord} it suffices to show that
$\dom{f}$ is transitive. So fix $\beta \in \dom{f}$, i.e.,
$\ordeq{\beta}{\tuple{B_b, \lessdot_b}}$ for some $b$. If $\gamma \in
\beta$, then $\gamma \in \dom{f}$ by
\olref[iso]{wellordinitialsegment}; generalising, $\beta \subseteq
\dom{f}$.
\end{proof}

This result licenses the following definition, which we have wanted to
offer since \olref[vn]{sec}:

\begin{defn}
If $\tuple{A, <}$ is a well-ordering, then its order type,
$\ordtype{A, <}$, is the unique ordinal $\alpha$ such that
$\ordeq{\tuple{A, < }}{\alpha}$.
\end{defn}

Moreover, this definition licenses two nice principles:

\begin{cor}\ollabel{ordtypesworklikeyouwant}
Where $\tuple{A, <}$ and $\tuple{B, \lessdot}$ are well-orderings: 
\begin{align*}
	\ordtype{A, <} = \ordtype{B, \lessdot}&\text{ iff }\ordeq{\tuple{A, <}}{\tuple{B, \lessdot}}\\
	\ordtype{A, <} \in \ordtype{B, \lessdot}&\text{ iff }\ordeq{\tuple{A, <}}{\tuple{B_b, \lessdot_b}}\text{ for some }b \in B
\end{align*}
\end{cor}

\begin{proof}
The identity holds by \olref[basic]{ordisoidentity}. To prove
the second claim, let $\ordtype{A, <} = \alpha$ and $\ordtype{B,
\lessdot} = \beta$, and let $f \colon \beta \to \tuple {B, \lessdot}$
be our isomorphism. Then:
\begin{align*}
	\alpha \in \beta&\text{ iff }\funrestrictionto{f}{\alpha} \colon \alpha \to B_{f(\alpha)}\text{ is an isomorphism}\\
	&\text{ iff }\ordeq{\tuple{A, <}}{\tuple{B_{f(\alpha)}, \lessdot_{f(\alpha)}}}\\
	&\text{ iff }\ordeq{\tuple{A, <}}{\tuple{B_b, \lessdot_b}}\text{ for some $b \in B$}
\end{align*}
by \olref[iso]{ordisounique},
\olref[iso]{wellordinitialsegment}, and
\olref[basic]{ordissetofsmallerord}.
\end{proof}

\end{document}