\documentclass[../../../include/open-logic-section]{subfiles}

\begin{document}

\olsection{sth}{story}{urelements}{Urelements or Not?}

In the next few chapters, we will try to extract axioms from the
cumulative-iterative conception of set. But, before going any further,
we need to say something more about \emph{urelements}. 

The picture of \olref[approach]{sec} allowed us only to form new sets
from old \emph{sets}. However, we might want to allow that certain
\emph{non-sets}---cows, pigs, grains of sand, or whatever---can be
!!{element}s of sets. In that case, we would start with certain basic
elements, \emph{urelements}, and then say that at each stage $S$ we
would form ``all possible'' sets consisting of urelements or sets
formed at stages before $S$ (in any combination). The resulting
picture would look more like this:
\begin{center}
  \olasset{\olpath/assets/tikzpictures/sets-theory/urelements.tikz}
\end{center}
So now we have a decision to take: \emph{Should we allow urelements?}

Philosophically, it makes sense to include urelements in our
theorising. The main reason for this is to make our set theory
\emph{applicable}. To illustrate the point, recall from
\olref[sfr][siz][]{chap} that we say that two sets $A$ and~$B$ have
the same size, i.e., $\cardeq{A}{B}$, iff there is a bijection between
them. Now, if the cows in the field and the pigs in the sty both form
sets, we can offer a set-theoretical treatment of the claim ``there
are as many cows as pigs''. But if we ban urelements, so that the cows
and the pigs do \emph{not} form sets, then that set-theoretical
treatment will be unavailable. Indeed, we will have no straightforward
ability to apply set theory to anything other than sets themselves.
(For more reasons to include urelements, see \cite[pp.~vi, 24,
50--1]{Potter2004}.)

Mathematically, however, it is quite rare to allow urelements. In
part, this is because it is \emph{very slightly} easier to formulate
set theory without urelements. But, occasionally, one finds more
interesting justifications for excluding urelement from set theory:
\begin{quote}
	In accordance with the belief that set theory is the foundation of
	mathematics, we should be able to capture all of mathematics by
	just talking about sets, so our variable should not range over
	objects like cows and pigs. 
	%But if $C$ is a cow, $\{C\}$ is a set, but not a legitimate mathematical object. 
	\cite[p.~8]{Kunen1980}
\end{quote}
So: a focus on applicability would suggest \emph{including}
urelements; a focus on a reductive foundational goal (reducing
mathematics to pure set theory) might suggest \emph{excluding} them.
Mild laziness, too, points in the direction of excluding urelements. 

We will follow the laziest path. Partly, though, there is a
pedagogical justification. Our aim is to introduce you to the elements
of set theory that you would need in order to get started on the
philosophy of set theory. And most of that philosophical literature
discusses set theories formulated \emph{without} urelements. So this
book will, perhaps, be of more use, if it hews fairly closely to that
literature.

\end{document}