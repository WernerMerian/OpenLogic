\documentclass[../../../include/open-logic-section]{subfiles}

\begin{document}

\olsection{sth}{z}{pairs}{Pairs}

The next axiom to consider is the following:

\begin{axiom}[Pairs]
For any sets $a, b$, the set $\{a, b\}$ exists.
\[
	\forall a \forall b \exists P \forall x (x \in P \liff (x = a \lor x = b))
\]
\end{axiom}

Here is how to justify this axiom, using the iterative conception. Suppose $a$ is available at stage $S$, and $b$ is available at stage $T$. Let $M$ be whichever of stages $S$ and $T$ comes later. Then since $a$ and $b$ are both available at stage $M$, the set $\{a,b\}$ is a possible collection available at any stage after $M$ (whichever is the greater).

But hold on!{} Why assume that there \emph{are} any stages after $M$? If there are none, then our justification will fail. So, to justify Pairs, we will have to add another principle to the story we told in \olref[sth][z][story]{sec}, namely:
\begin{enumerate}
	\item[] \stagessucc. There is no last stage.
\end{enumerate}
Is this principle justified? Nothing in Shoenfield's story stated
\emph{explicitly} that there is no last stage. Still, even if it is
(strictly speaking) an extra addition to our story, it fits well with
the basic idea that sets are formed in stages. We will simply accept
it in what follows. And so, we will accept the Axiom of Pairs too.

Armed with this new Axiom, we can prove the existence of plenty more sets. For example:

\begin{prop}\ollabel{prop:pairsconsequences}
For any sets $a$ and $b$, the following sets exist:
	\begin{enumerate}
		\item\ollabel{singleton} $\{a\}$
		\item\ollabel{binunion} $a \cup b$
		\item\ollabel{tuples} $\tuple{a, b}$
	\end{enumerate}
\end{prop}

\begin{proof}
\olref{singleton}. By Pairs, $\{a, a\}$ exists, which is $\{a\}$ by
Extensionality.

\olref{binunion}. By Pairs, $\{a, b\}$ exists. Now $a \cup b = \bigcup
\{a, b\}$ exists by Union.

\olref{tuples}. By \olref{singleton}, $\{a\}$ exists. By Pairs, $\{a,
b\}$ exists. Now $\{\{a\}, \{a, b\}\} = \tuple{a, b}$ exists, by Pairs
again.
\end{proof}

\begin{prob}
Show that, for any sets $a, b, c$, the set $\{a, b, c\}$ exists.
\end{prob}

\begin{prob}
Show that, for any sets $a_1, \ldots, a_n$, the set $\{a_1, \ldots,
a_n\}$ exists.
\end{prob}

%\begin{proof} By two applications of Pairs, $\{\{a_1, a_2\}, \{a_1,
%   a_3\}\}$ exists. By Union and Extensionality, $\bigcup \{\{a_1,
%   a_2\}, \{a_1, a_3\}\} = \{a_1, a_2, a_3\}$ exists. Repeat this
%   trick as often as necessary. \end{proof}

\end{document}
