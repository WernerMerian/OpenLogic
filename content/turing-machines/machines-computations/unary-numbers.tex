% Part: turing-machines
% Chapter: machines-computations
% Section: unary-numbers

\documentclass[../../../include/open-logic-section]{subfiles}

\begin{document}

\olsection{tur}{mac}{una}{Unary Representation of Numbers}

\begin{explain}
Turing machines work on sequences of symbols written on their tape.
Depending on the alphabet a Turing machine uses, these sequences of
symbols can represent various inputs and outputs.  Of particular
interest, of course, are Turing machines which compute
\emph{arithmetical} functions, i.e., functions of natural numbers.  A
simple way to represent positive integers is by coding them as
sequences of a single symbol~$\TMstroke$.  If $n \in \Nat$, let
$\TMstroke^n$ be the empty sequence if $n = 0$, and otherwise the
sequence consisting of exactly $n$ $\TMstroke$'s.
\end{explain}

\begin{defn}[Computation]
A Turing machine~$M$ \emph{computes} the function $f\colon \Nat^k \to \Nat$ iff
$M$~halts on input
\[
\TMstroke^{n_1} \TMblank \TMstroke^{n_2} \TMblank \dots \TMblank \TMstroke^{n_k}
\]
with output $\TMstroke^{f(n_1, \dots, n_k)}$.
\end{defn}

\begin{prob}
  Give a definition for when a Turing machine~$M$ computes the
  function $f\colon \Nat^k \to \Nat^m$.
\end{prob}

\begin{ex}\ollabel{ex:adder}
\emph{Addition:}
Let's build a machine that computes the function $f(n,m) = n + m$.
This requires a machine that starts with two blocks of $\TMstroke$'s
of length $n$ and~$m$ on the tape, and halts with one block consisting
of $n+m$~$\TMstroke$'s. The two input blocks of~$\TMstroke$'s are
separated by a~$\TMblank$, so one method would be to write a stroke on
the square containing the $\TMblank$, and erase the last~$\TMstroke$.
\begin{figure}
\centering
\olasset{\olpath/assets/tikzpictures/turing-machines/adder.tikz}
\caption{A machine computing $f(x,y) = x+y$}
\ollabel{fig:adder}
\end{figure}
\end{ex}

\begin{prob}
Trace through the configurations of the machine from
\olref[tur][mac][una]{ex:adder} for input~$\tuple{3,2}$. What happens
if the machine computes $0+0$?
\end{prob}

\begin{explain}
In \olref[rep]{ex:doubler}, we gave an example of a Turing machine
that takes as input a sequence of~$\TMstroke$'s and halts with a sequence
of twice as many~$\TMstroke$'s on the tape---the doubler machine.
However, because the output contains $\TMblank$'s to the left of the
doubled block of~$\TMstroke$'s, it does not actually compute the
function $f(x) = 2x$, as you might have assumed. We'll describe two
ways of fixing that.
\end{explain}

\begin{ex}
The machine in \olref{fig:doubler-disc} computes the function $f(x) =
2x$. Instead of erasing the input and writing two $\TMstroke$'s at the
far right for every $\TMstroke$ in the input as the machine from
\olref[rep]{ex:doubler} does, this machine adds a single~$\TMstroke$
to the right for every~$\TMstroke$ in the input. It has to keep track
of where the input ends, so it leaves a~$\TMblank$ between the input
and the added strokes, which it fills with a~$\TMstroke$ at the very
end. And we have to ``remember'' where we are in the input, so we
temporarily replace a~$\TMstroke$ in the input block by a~$\TMblank$.
\begin{figure}
\centering
\olasset{\olpath/assets/tikzpictures/turing-machines/doubler-disc.tikz}
\caption{A machine computing $f(x) = 2x$}
\ollabel{fig:doubler-disc}
\end{figure}
\end{ex}

\begin{ex}\ollabel{ex:mover}
A second possibility for computing $f(x) = 2x$ is to keep the original
doubler machine, but add states and instructions at the end which move
the doubled block of strokes to the far left of the tape.  The machine
in \olref{fig:mover} does just this last part: started on a tape
consisting of a block of~$\TMblank$'s followed by a block
of~$\TMstroke$'s (and the head positioned anywhere in the block
of~$\TMblank$'s), it erases the $\TMstroke$'s one at a time and writes
them at the beginning of the tape. In order to be able to tell when it
is done, it first marks the end of the block of $\TMstroke$'s with
a~$\TMendtape$ symbol, which gets deleted at the end. We've started
numbering the states at~$q_6$, so they can be added to the doubler
machine. All you'll need is an additional instruction $\delta(q_5,
\TMblank) = \tuple{q_6,\TMblank,\TMstay}$, i.e., an arrow from~$q_5$
to~$q_6$ labelled~$\TMtrans{\TMblank}{\TMblank}{\TMstay}$. (There is
one subtle problem: the resulting machine does not work for
input~$x=0$. We'll leave this as an exercise.)
\begin{figure}
\centering
\olasset{\olpath/assets/tikzpictures/turing-machines/mover.tikz}
\caption{Moving a block of $\TMstroke$'s to the left}
\ollabel{fig:mover}
\end{figure}
\end{ex}

\begin{prob}
In \olref[tur][mac][una]{ex:mover} we described a machine consisting
of a combination of the doubler machine from
\olref[tur][mac][una]{fig:doubler-disc} and the mover machine from
\olref[tur][mac][una]{fig:mover}. What happens if you start this
combined machine on input~$x=0$, i.e., on an empty tape?  How would
you fix the machine so that in this case the machine halts with
output~$2x=0$? (You should be able to do this by adding one state and
one transition.)
\end{prob}

\begin{prob}
\emph{Subtraction:} Design a Turing machine that when given an input
of two non-empty strings of strokes of length $n$ and~$m$, where $n >
m$, computes the function $f(n,m) = n - m$.
\end{prob}

\begin{prob}
\emph{Equality:} Design a Turing machine to compute the following function:
\[
\fn{equality}(n,m) = 
\begin{cases}
  \text{1} & \text{if~$n = m$} \\
  \text{0} & \text{if~$n \neq m$}
\end{cases}
\]
where~$n$ and~$m \in \PosInt$.
\end{prob}

\begin{prob}
Design a Turing machine to compute the function $\min(x,y)$ where $x$
and $y$ are positive integers represented on the tape by strings of
$\TMstroke$'s separated by a $\TMblank$. You may use additional
symbols in the alphabet of the machine.

The function $\min$ selects the smallest value from its arguments, so
$\min(3,5)=3$, $\min(20,16)=16$, and $\min(4,4)=4$, and so on.
\end{prob}

\begin{defn}
  A Turing machine~$M$ computes the partial function $f\colon \Nat^k
  \pto \Nat$ iff, 
  \begin{enumerate}
    \item $M$ halts on input $\TMstroke^{n_1}\concat\TMblank\concat
    \dots \concat\TMblank\concat\TMstroke^{n_k}$ with output $\TMstroke^{m}$ if $f(n_1, \dots, n_k) = m$.
    \item $M$ does not halt at all, or with an output that is not a
    single block of~$\TMstroke$'s if $f(n_1, \dots, n_k)$ is undefined.
  \end{enumerate}
\end{defn}

\end{document}
