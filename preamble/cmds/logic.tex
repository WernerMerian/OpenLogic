% Typesetting commands for logical concepts
% =========================================

% In order to unformly typeset certain types of symbols uniformly, the
% OLP texts use special commands which carry out the
% typesetting. Thus, e.g., all structures, which appear in the texts,
% say, as `\Struct{M}`, can be typeset according to preference. By
% default, they are typeset in fraktur (e.g., $\mathfrak{M}$) but this
% can be configured by changing the definition of the `\Struct`
% command.  Note that often the default behavior is to only apply the
% typeface command to the first token in the argument, e.g., `\Struct
% M_n` will generate $\mathfrak M_n$ and not $\mathfrak{M_n}$.

% - `\Struct{M}` - First-order structures; by default, the first token
% in Fraktur

% `\applytofirst` will apply #1 to #2 after expanding #2 once.

\def\applytofirst#1#2{{\expandafter#1#2}}

\DeclareDocumentCommand \Struct { m }{\applytofirst{\mathfrak}{#1}}

% - `\Lang{L}` - Languages; default: set first token in callgraphic
% font

\DeclareDocumentCommand \Lang { m }{\applytofirst{\mathcal}{#1}}

% - `\Log{L}` - Logics; default: set entirely in boldface

\DeclareDocumentCommand \Log { m o }{\ensuremath{\mathbf{#1}
		\IfNoValueTF {#2}{}{_{#2}}}}

% - Some logics

\DeclareDocumentMacro {\LogCL} {\Log{C}}
\DeclareDocumentMacro {\LogIL} {\Log{I}}
\DeclareDocumentMacro {\LogLuk} {\Log{\textbf{\L}}}
\DeclareDocumentMacro {\LogGod} {\Log{G}}
\DeclareDocumentMacro {\LogKs} {\Log{Ks}}
\DeclareDocumentMacro {\LogKw} {\Log{Kw}}
\DeclareDocumentMacro {\LogLP} {\Log{LP}}
\DeclareDocumentMacro {\LogRM} {\Log{RM}}
\DeclareDocumentMacro {\LogHal} {\Log{Hal}}


% - `\Obj` - Object-language symbols; default: set entirely in
% sans-serif italics

\DeclareDocumentCommand \Obj { m }{\mathsfit{#1}}

%  - `\Term{f}{t_1, t_2)` - Term; default produces
%  function symbol followed by arguments surrounded by
%  parentheses. Some prefer no parentheses around the arguments.

\DeclareDocumentCommand \Term { m m }{ \mathord{#1}(#2) }

%  - `\Atom{R}{t_1, t_2)` - Atomic formula; default produces
%  predicate symbol followed by arguments surrounded by
%  parentheses. Some prefer no parentheses around the arguments.

\DeclareDocumentCommand \Atom { m m }{ \mathord{#1}(#2) }

% - `\Ax{A}`: typeset an axiom

\DeclareDocumentCommand \Ax { m } {\ensuremath{\mathrm{#1}}}

% - `\PIso` - Set of all partial isomorphisms

\DeclareDocumentCommand \PIso { m }{\mathcal{#1}}

% - `\fn{func}` -- typeset a function name

\DeclareDocumentCommand \fn { m } {\mathrm{#1}}

% - `\Th{T}` -- typeset name of a theory

\DeclareDocumentCommand \Th { m } {\mathbf{#1}}