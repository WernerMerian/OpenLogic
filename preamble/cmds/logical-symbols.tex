% Logical symbols
% ---------------

% The following commands are used in the OLP texts for logical
% symbols. Their definitions can be customized to produce different
% output.

% ### Truth Values
%
% - `\True` defaults to $\mathbb{T}$ and `\False` to $\mathbb{F}$.

\DeclareDocumentMacro \True {\ensuremath{\mathbb{T}}}
\DeclareDocumentMacro \False {\ensuremath{\mathbb{F}}}

% Other truth values

\DeclareDocumentMacro \Indet {\ensuremath{\mathbb{I}}}
\DeclareDocumentMacro \Undef {\ensuremath{\mathbb{U}}}

% ### Propositional Constants and Connectives
%
% - Falsity is `\lfalse` and defaults to $\bot$.

\DeclareDocumentMacro \lfalse {\bot}

% - Truth is `\ltrue` and defaults to $\top$.

\DeclareDocumentMacro \ltrue {\top}

% - Negation is `\lnot` and defaults to $\lnot$.  To use a different
% symbol (e.g., tilde), use the following line.

% `\DeclareDocumentMacro \lnot {\mathord{\sim}}`

% - Conjunction is `\land` and deaults to $\land$.  to use ampersand,
% uncomment the following line

% `\DeclareDocumentMacro \land {\mathbin{\&}}`

% - Disjunction is `\lor` and defaults to $\lor$.

% - Conditional is `\lif` and defaults to $\rightarrow$.  To use a
% different symbol, replace `\rightarrow` in the definition, e.g., by
% `\supset`

\DeclareDocumentMacro \lif {\mathbin{\rightarrow}}

% - The biconditional is `\liff` and defaults to $\leftrightarrow$.  To
% use the triple bar $\equiv$ replace with `\equiv`.

\DeclareDocumentMacro \liff {\mathbin{\leftrightarrow}}

% - The conditional `\cif` and defaults to `\boxright` which produces
% - Lewis's box-arrow symbol.

\DeclareDocumentMacro \cif {\boxright}

% - The strict conditional `\strictif`

\DeclareDocumentMacro \strictif {\fishhookright}

% Quantifiers 
% ----------- 

% The quantifier symbols are provided as commands `\lexists` and
% `\lforall` which take two optional arguments. If no arguments are
% provided, it they just typeset the quantifier symbol. With one
% optional argument they produce the quantifier together with a
% variable, and this may include parenthesesaround the quantifier and
% variable. The second optional argument producesthe
% quantifier/variable combination plus the formula in the scope of the
% formula with appropriate spacing.  For instance,
% `\lexists[x][!A(x)]` will, by default, produce $\exists
% x\,\varphi(x)$.

% - The existential quantifier is `\lexists`.  Replace `\exists` with
% `\boldsymbol{\exists}` for boldface, or redefine appropriately if
% you want parentheses around $\exists x$.

\DeclareDocumentCommand \lexists { t{!} o o } {
	\exists 
	\IfBooleanTF {#1} 
	\mathexclaim    % unique
	\relax % not unique
	\IfNoValueTF {#2} 
	\relax     % no arguments
	{ #2 } % one argument: variable
	\IfNoValueTF {#3} 
	\relax
	{ \, #3 }      % two arguments: space and matrix
}

% - The universal quantifier is `\lforall`.

\DeclareDocumentCommand \lforall { o o } {
	\IfNoValueTF {#1} 
	{ \forall }    % no arguments
	{ \forall #1 } % one argument: variable
	\IfNoValueTF {#2} 
	\relax
	{ \, #2 }      % two arguments: space and matrix
}

% - The identity relation is also provided as `\eq`. By itself, it
% produces the identity reation symbol (default: $=$) by itself. With
% two optional arguments, it typesets the corresponding atomic
% formula, e.g., `\eq[x][y]` produces $x = y$.  `\eq/` produces the
% negated symbol (formula).

\DeclareDocumentCommand \eq { t{/} o o } {
	\IfNoValueTF {#3}
	% no optional arguments: just typeset symbol
	{ \IfBooleanTF{#1}{ \neq }{ = } }
	% optional arguments: typeset atomic formula
	{ \IfBooleanTF{#1}{ #2 \neq #3}{#2 = #3} }
}