% Metalogical Relations
% ---------------------
%
% Metalogical relationships, such as truth in a structure, validity,
% consequence, and provability, are also provided as commands. Uniform
% use of these commandsinstead of hard-coded typesetting according to
% specific conventions guarantees that by changing the definitions
% below you can uniformly change notation in the text.


% ### Substitution

% -`\subst{t}{x}`: typeset the substitution notation

\DeclareDocumentCommand \subst { m m } {#1/#2}

% - `\SSubst{A}{s}`: typeset simultaneous substitution (expects $s$ to
% be a list of `\subst{t}{x}` expressions, say)

\DeclareDocumentCommand \SSubst { m m } {
	#1[#2]}

% - `\Subst{!A}{t}{x}`: The operation of substituting a term for a
% (free) variable in another term or in a formula.  The default is
% $\varphi[x/t]$, other common notations are $\varphi^t_x$,
% $\varphi\{t \rightarrow x\}$, or $S^t_x \varphi$.

\DeclareDocumentCommand \Subst { m m m } {
	#1[\subst{#2}{#3}]}

% ### pre-Substitution

\DeclareDocumentCommand \pSubst { m m m } {
	#1[#2/#3]^{-}
}

% ### The satisfaction/truth relation 

% - `\Sat[/]{M}{!A}[s]`, the relation of being satisfied in a
% structure (relative to an assignment), is provided as the command
% `\Sat` with two mandatory arguents (the structure and the formula)
% and one optional argument (the assignment).  Use `\Sat/` to create
% the negated relation.  By default, `\Sat{M}{!A}[s]` is typeset as
% $\mathfrak{M}, s \models \varphi$.

\DeclareDocumentCommand \Sat { t{/} m m o } {
	\IfBooleanTF{#1}{
		% negated
		\IfNoValueTF {#4} 
		{ \Struct{#2} \nvDash #3 }
		{ \Struct{#2}, #4 \nvDash #3}}{
		% not negated
		\IfNoValueTF {#4} 
		{ \Struct{#2} \vDash #3 }
		{ \Struct{#2}, #4 \vDash #3 }}
}

% ### The derivability relation

% `\Proves[L]` is used to create the symbol for the derivability
% relation, `\Proves/` for the negation. By default this creates
% $\vdash$; e.g., `\Gamma \Proves !A` yields $\Gamma \vdash
% \varphi$. An optional argument may be used for the calculus or logic
% relative to which the provability relation is defined; by default it
% creates a subscript on the turnstile.

\DeclareDocumentCommand \Proves { t{/} o } {
	\IfBooleanTF {#1}{
		\IfNoValueTF {#2} 
		{ \nvdash }
		{ \nvdash_{#2} }}{
		\IfNoValueTF {#2} 
		{ \vdash }
		{ \vdash_{#2} }}
}

% - `\Thms{X}`: theorems a set of formulas

\DeclareDocumentCommand \Thms { m } {\mathrm{Thm}(#1)}

% - `\PAx`: the set of propositional axioms

\DeclareDocumentMacro \PAx { \mathrm{Ax}_0 }

% ### The semantic consequence relation relation

% `\Entails` is the semantic counterpart of `\Proves` and defaults to
% $\vDash$. It also takes an optional `/` for $\nvDash$ and an
% optional argument for a subscript.

\DeclareDocumentCommand \Entails { t{/} o } {
	\IfBooleanTF {#1}{
		\IfNoValueTF {#2} 
		{ \nvDash }
		{ \nvDash_{#2} }}{
		\IfNoValueTF {#2} 
		{ \vDash }
		{ \vDash_{#2} }}
}


% ### Model-theoretic notions and symbols

% - `\Domain{M}` - domain of a structure, e.g., `\Domain{M}` gives
% $\left|\mathfrak M\right|$.

\DeclareDocumentCommand \Domain { m }{\left| \Struct{#1} \right|}

% - `\Assign{R}{M}` - Assignment (value of) of a constant/predicate symbol
% in a structure; e.g., `\Assign{R}{M}` produces $R^\mathfrak{M}$.

\DeclareDocumentCommand \Assign { m m }{\mathord{#1^{\Struct{#2}}}}

% - `\varAssign{s'}{s}{x}[o]` - Assignment variant. Takes three mandatory
% argument (s' differs from s at most at x) and one optional one (s' assigns
% o to x. Default: `\varAssign{s'}{s}{x}` produces `s' \sim_{x} s`
% and `\varAssign{s'}{s}{x}[o] produces `s' = s[o/x]`.

\DeclareDocumentCommand \varAssign { m m m o } {
	\IfNoValueTF {#4}
	% optional argument not present
	{ #1 \sim_{#3} #2 }
	% optional argument present
	{ #1 = #2[^{#4}/{#3}] }
}

% - `\Value{t}{M}[s]` - Value of a term in a structure. Takes two mandatory
% arguments (term and structure) and one optional argument (variable
% assignment). By default, `\Value{t}{M}[s]` produces 
% $\mathrm{Val}^\mathfrak{M}_s(t)$.

\DeclareDocumentCommand \Value { m m o} {
	\IfNoValueTF {#3}
	% optional argument not present
	{ \mathrm{Val}^{\Struct{#2}}(#1) }
	% optional argument present
	{ \mathrm{Val}^{\Struct{#2}}_{#3}(#1) }
}

% - `\pAssign{v}` - Typeset a truth-value assignment

\DeclareDocumentCommand \pAssign { m } {\applytofirst{\mathfrak}{#1}}

% - `\pValue{v}(A)[L]` - Truth value of a formula under a truth-value assignment.

\DeclareDocumentCommand \pValue { m d() o}{
	\overline{\pAssign{#1}}%
	\IfNoValueTF{#3}{}{_{#3}}%
	\IfNoValueTF {#2}{}{(#2)}
}

% - `\pSat[/]{v}{!A}[L]`, the relation of being satisfied by a
% truth-value assignment in a logic L.

\DeclareDocumentCommand \pSat { t{/} m m o } {
	\pAssign{#2} 
	\IfBooleanTF{#1}{\nvDash}{\vDash}%
	\IfNoValueTF{#4}{}{_{#4}}
	#3
}

% - `\tf{\star}[L]`: truth function for $\star$ in $\mathbf{L}$

\DeclareDocumentCommand \tf { m o } {
	\widetilde{#1}%
	\IfNoValueTF{#2}{}{_{#2}}
}


% - `\substruct`: symbol for the substructure relation

\DeclareDocumentMacro \substruct {\subseteq}

% - `\Theory{M}`: theory of a structure

\DeclareDocumentCommand \Theory { m } {\mathrm{Th}(\Struct{#1})}

% - `\Mod[L](L'){T}`: class of models of a theory/sentence $T$ in a
% language $\mathcal{L}$ and logic $L'$.

\DeclareDocumentCommand \Mod { o d() m } {
	\IfNoValueTF {#2} {
		% optional logic argument not present
		\IfNoValueTF {#1}{
			\mathrm{Mod}(#3) }{
			\mathrm{Mod}^{\Lang{#1}}(#3) }}{
		% optional logic argument present
		\IfNoValueTF {#1}{
			\mathrm{Mod}_{#2}(#3)}{
			\mathrm{Mod}_{#2}^{\Lang{#1}}(#3)}}
}


% - `\elemequiv`: elementary equivalence (infix relation)

\DeclareDocumentCommand \elemequiv { t{/} o } {
	\IfBooleanTF {#1}{
		\IfNoValueTF {#2} 
		{ \not\equiv }
		{ \not\equiv_{#2} }}{
		\IfNoValueTF {#2} 
		{ \equiv }
		{ \equiv_{#2} }}
}

% - `\eqc`: the equivalence class the element (first argument) belongs
% to; second argument is used to mark the equivalence relation if
% there's more than one

\DeclareDocumentCommand \eqc { m o } {
	\IfNoValueTF {#2}
	{[#1]}
	{[#1]_{#2}}
}

% - `\rep`: the representative of an equivalence class, the second
% argument is used to mark the equivalence relation if there's more
% than one

\DeclareDocumentCommand \rep { m o } {
	\IfNoValueTF {#2}
	{\underline{#1}}
	{{\underline{#1}}_{#2}}
}

% - `\iso[/][p]`: relation of being (partially) isomorphic

\DeclareDocumentCommand \iso { t{/} o } {
	\IfBooleanTF {#1}{
		\IfNoValueTF {#2} 
		{ \not\simeq }
		{ \not\simeq_{#2} }}{
		\IfNoValueTF {#2} 
		{ \simeq }
		{ \simeq_{#2} }}
}



% - `\ident`: syntactic identity between expressions (infix relation),

\DeclareDocumentMacro \ident {\equiv}

% - `\QuantRank{!A}`: quantifier rank of a formula

\DeclareDocumentCommand \QuantRank { m } {\mathrm{qr}(#1)}

% - `\Expan{M}{R}`: expansion of a structure by a relation (etc.)

\DeclareDocumentCommand \Expan { m m } {(\Struct{#1}, #2)}

% `\nssucc`, `\nsplus`, `\nstimes`, `\nsless`: non-standard
% arithmetical operations

\DeclareDocumentMacro \nszero {\mathbf{z}}
\DeclareDocumentMacro \nssucc {*}
\DeclareDocumentMacro \nsplus {\oplus}
\DeclareDocumentMacro \nstimes {\otimes}
\RequirePackage{stmaryrd}
\DeclareDocumentMacro \nsless {\varolessthan}