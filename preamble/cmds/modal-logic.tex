% Modal logic
% ===========

% - `\mModel{M}` - modal structures; default: set first token in
% Fraktur

\DeclareDocumentCommand \mModel { m }{\applytofirst{\mathfrak}{#1}}

% `\mSat[/]{M}{!A}[w]`, the relation of being satisfied in a
% model (at a world), is provided as the command
% `\mSat` with two mandatory arguments (the model and the formula)
% and one optional argument (the world).  Use `\mSat/` to create
% the negated relation.  By default, `\mSat{M}{!A}[w]` is typeset as
% $\mathfrak{M}, w \models \varphi$.

\DeclareDocumentCommand \mSat { t{/} m m o } {%
	\IfBooleanTF{#1}{%
		% negated
		\IfNoValueTF {#4} 
		{ \mModel{#2} \nVdash #3 }
		{ \mModel{#2}, #4 \nVdash #3}}{%
		% not negated
		\IfNoValueTF {#4} 
		{ \mModel{#2} \Vdash #3 }
		{ \mModel{#2}, #4 \Vdash #3 }}}

% - `\mClass{C}` --- typeset class of models

\DeclareDocumentCommand \mClass { m }{\mathcal{#1}}

% - `\Nec`: produces abbreviation for Necessitation.

\DeclareDocumentMacro \Nec {\textsc{nec}}

% - `\RK`: produces abbreviation for Rule K

\DeclareDocumentMacro \RK {\textsc{rk}}

% - `\Dual`: produces abbreviation for Dual

\DeclareDocumentMacro \Dual {\textsc{dual}}

% - `\Taut`: produces abbreviation for Dual

\DeclareDocumentMacro \Taut {\textsc{taut}}

% - `\PL`: produces abbreviation for ``Propositional Logic''

\DeclareDocumentMacro \PL {\textsc{pl}}

% - `\Prop{M}{A}`: the proposition defined by $A$ in $\mathfrak{M}$

\DeclareDocumentCommand \Prop { m m } {
	{[\!\![} #2 {]\!\!]_{\mModel{#1}}}
}

% - `\ST`: The standard translation

\DeclareDocumentMacro \ST {\mathord{\mathrm{ST}}}

% - TikZ style for modal models

\tikzset{
	modal/.style={>=stealth',
		shorten >=1pt,
		shorten <=1pt,
		auto,
		node distance=1.5cm,
		label distance=2pt,
		semithick},
	every label/.style={phantom,align=left},
	world/.style = {circle,draw,minimum size=0.5cm,fill=gray!15},
	modal every node/.style={world},
	point/.style={circle,draw,inner sep=0.5mm,fill=black},
	phantom/.style={rectangle,inner sep=0pt,draw=none,fill=none},
	reflexive above/.style={->,loop,looseness=7,in=60,out=120},
	reflexive below/.style={->,loop,looseness=7,in=240,out=300},
	reflexive left/.style={->,loop,looseness=7,in=150,out=210},
	reflexive right/.style={->,loop,looseness=7,in=30,out=330}
}

\DeclareDocumentCommand \mTrue { m }{\ensuremath{#1}}
\DeclareDocumentCommand \mFalse { m }{\ensuremath{\lnot #1}}