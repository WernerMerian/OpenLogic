% Proofs and Derivations
% ----------------------

% - The sequent symbol `\Sequent` produces $\Rightarrow$ by
% default. Change the definition for $\vdash$, or another symbol.

\DeclareDocumentMacro \Sequent {\Rightarrow}

\DeclareDocumentMacro \nSequent {\mid}

% The sequent symbol in proofs displays as the above sequent symbol.

\DeclareDocumentMacro \fCenter {\ensuremath{\,\Sequent\,}}

% - Rule names: `\LeftR{Op}` typesets the name of a left rule for
% operator `Op`, e.g., `\LeftR{\land}` produces `$\land$L`.
% `\RightR{Op}` does the same for right rules.

\DeclareDocumentCommand \LeftR { m } {\ensuremath{{#1}\mathrm{L}}}

\DeclareDocumentCommand \RightR { m } {\ensuremath{{#1}\mathrm{R}}}

\DeclareDocumentCommand \iR { m m o} {\ensuremath{{#1\IfNoValueTF{#3}{}{_{#3}}}{#2}}}

% - `\Weakening`: produces name or abbreviation for weakening rule,
% e.g., ``W''.

\DeclareDocumentMacro \Weakening {\text{W}}

% - `\Contraction`: produces name or abbreviation for contraction rule,
% e.g., ``C''.

\DeclareDocumentMacro \Contraction {\text{C}}

% - `\Exchange`: produces name or abbreviation for exchange rule,
% e.g., ``X''.

\DeclareDocumentMacro \Exchange {\text{X}}

% - `\Cut`: produces name or abbreviation for cut rule,
% e.g., ``Cut''.

\DeclareDocumentMacro \Cut {\text{Cut}}

% - Rule names: `\Intro{Op}` typesets the name of an intro rule for
% operator `Op`, e.g., `\Intro{\land}` produces `$\land$Intro`.
% `\Elim{Op}` does the same for elimination rules.

\DeclareDocumentCommand \Intro { m } {\ensuremath{{#1}\mathrm{Intro}}}

\DeclareDocumentCommand \Elim { m } {\ensuremath{{#1}\mathrm{Elim}}}

% - `\FalseInt`, `\FalseCl`: produces name or abbreviation for
% intuitionistic and classical absurdity rule, e.g., ``$\bot_I$,''
% ``$\bot_C$''.

\DeclareDocumentMacro \FalseInt {\ensuremath{\lfalse_I}}
\DeclareDocumentMacro \FalseCl {\ensuremath{\lfalse_C}}

% - `\Discharge{!A}{n}`: typesets a discharged assumption with label
% $n$, e.g., $[!A]^n$.

\DeclareDocumentCommand \Discharge { m m }{[#1]^{#2}}

% - `\DischargeRule{Rule}{n}`: used in a `prooftree` environment to
% provide the labels for an inference that discharges an assumption.

\DeclareDocumentCommand \DischargeRule { m m }{
	\RightLabel{#1}
	\LeftLabel{\scriptsize $#2$}
}