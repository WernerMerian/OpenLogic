% Additional set theory stuff
% ================
% From Tim Button's Open Set Theory

\DeclareDocumentMacro \unitline {\text{L}}
\DeclareDocumentMacro \unitsquare {\text{S}}
\newcommand{\onesphere}{\mathbf{S}}
\newcommand\rotationsgroup{R}

\DeclareDocumentCommand \cardneq { m m } {#1 \not\approx #2}
\DeclareDocumentCommand \cardnless { m m } {#1 \npreceq #2}
\DeclareDocumentCommand \ordeq { m m } {#1 \cong #2}
\DeclareDocumentCommand \ordneq { m m } {#1 \ncong #2}
\DeclareDocumentCommand \funimage { m m } {#1[#2]}

\newcommand\stageshier{\emph{Stages-are-key}}
\newcommand\stagesord{\emph{Stages-are-ordered}}
\newcommand\stagesacc{\emph{Stages-accumulate}}
\newcommand\stagessucc{\emph{Stages-keep-going}}
\newcommand\stagesinf{\emph{Stages-hit-infinity}}
\newcommand\limofsize{\emph{Limitation-of-size}}
\newcommand\stagesinex{\emph{Stages-are-inexhaustible}}
\newcommand\stagescofin{\emph{Stages-are-super-cofinal}}

\usetikzlibrary{lindenmayersystems}
\pgfdeclarelindenmayersystem{Hilbert curve}{
	\rule{L -> +RF-LFL-FR+}
	\rule{R -> -LF+RFR+FL-}}

\newcommand\closureofunder[2]{\mathrm{clo}_{#1}(#2)}
\newcommand\Closureofunder[2]{\mathrm{Clo}_{#1}(#2)}
\newcommand\equivrep[2]{[#1]_{#2}}
\newcommand\equivclass[2]{#1/_{\!{#2}}}
\newcommand\Intequiv{\sim}
\newcommand\Ratequiv{\backsim}
\newcommand\Realequiv{\Bumpeq}
\newcommand\funrestrictionto[2]{#1\mathord{\restriction}_{#2}}
\newcommand\isomorphic{\cong}
\newcommand\nisomorphic{\ncong}
\newcommand\precdot{\mathrel{\prec{\mkern -12mu \cdot}}}
\newcommand\disjointsum{\sqcup}
\newcommand\ordtype[1]{\mathrm{ord}(#1)}
\newcommand\ordsucc[1]{#1^{+}}
\newcommand\cardsucc[1]{#1^{\oplus}}
\newcommand\rlexless{\mathrel{\sphericalangle}}
\newcommand\canonord\lhd
\newcommand\ordplus{+}
\newcommand\ordtimes{\cdot}
\newcommand\ordexpo[2]{#1^{(#2)}}
\newcommand\cardplus{\oplus}
\newcommand\cardtimes{\otimes}
\newcommand\cardexpo[2]{#1^{#2}}
\newcommand\funfromto[2]{{}^{#1}{#2}}
\newcommand\setrank[1]{\mathrm{rank}(#1)}
\DeclareMathOperator*{\supstrict}{\mathrm{lsub}}
\newcommand\trcl[1]{\mathrm{trcl}(#1)}
\newcommand\ZF{\Th{ZF}}
\newcommand\SP{\Th{SP}}
\newcommand\ZFC{\Th{ZFC}}
\newcommand\Z{\Th{Z}}
\newcommand\ZFminus{\ZF^{-}}
\newcommand\Zminus{\Z^{-}}
\newcommand\Zr{\Th{Zr}}
\newcommand\LT{\Th{LT}}

\newcommand\cardfont[1]{\mathfrak{#1}}