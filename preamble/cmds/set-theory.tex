% Special Sets and Mathematical Symbols
% -------------------------------------

% ### Set-theoretic operators

% - Set abstracts: Use `\Setabs{x}{!A(x)}` to produce the set abstract
% $\{ x : \varphi(x) \}$. If you prefer a $\mid$ to :, change the
% definition accordingly.

\DeclareDocumentCommand \Setabs { m m }{\{ #1 : #2 \}}

% - Fregean extensions: Use `\fregeext{x}{!A(x)}` to produce
%   $\epsilon x\, !A(x)$.

\DeclareDocumentCommand \fregeext { m m }{\oldepsilon #1 \, #2 }

% - Fregean number: Use `\fregenum{x}{!A(x)}` to produce 
%   $\# x\, !A(x)$.

\DeclareDocumentCommand \fregenum { m m }{\# #1 \, #2 }


% - `\Pow{X}`: Power set, produces $\wp(X)$

\DeclareDocumentCommand \Pow { m }{\wp(#1)}

% - `\dom{f}`: domain of a function

\DeclareDocumentCommand \dom { m }{\fn{dom}(#1)}

% - `\ran{f}`: range of a function
\DeclareDocumentCommand \ran { m }{\fn{ran}(#1)}

% - `\len{s}`: length of a sequence
\DeclareDocumentCommand \len { m }{\fn{len}(#1)}

% - `\emptyseq`: the empty sequence
\DeclareDocumentMacro \emptyseq {\Lambda}

% - `\restrict`: restriction of a function to a set (infix operator)
\DeclareDocumentMacro \restrict {\upharpoonright}

% - `\Complement{X}`: complement of a set
\DeclareDocumentCommand \Complement { m } {\overline{#1}}

% - `\card{X}`: cardinality of a set
\DeclareDocumentCommand \card { m } {\left| #1 \right|}

% - `\cardle{X}{Y}`: X is no larger than Y
\DeclareDocumentCommand \cardle { m m } {#1 \preceq #2}

% - `\cardless{X}{Y}`: X is smaller than Y
\DeclareDocumentCommand \cardless { m m } {#1 \prec #2}

% - `\cardeq{X}{Y}`: X is equinumerous with Y
\DeclareDocumentCommand \cardeq { m m } {#1 \approx #2}

% - `\tuple{x,y}`: pairs, tuples, sequences
\DeclareDocumentMacro \openTuple {\langle}
\DeclareDocumentMacro \closeTuple {\rangle}
\DeclareDocumentCommand \tuple { m } {\openTuple #1 \closeTuple}

% - `\comp{f}{g}`: composition of f with g, defaults to $g \circ f$
\DeclareDocumentCommand \comp { m m }{#2 \circ #1}

% - `\pto`: partial function arrow

\DeclareDocumentMacro \pto {\mathrel{\ooalign{\hfil$\mapstochar\mkern
			5mu$\hfil\cr$\to$}}}

% - `\fdefined`, `\fundefined`: postfix for defined, undefined
% functions

\DeclareDocumentMacro \fdefined {\downarrow}
\DeclareDocumentMacro \fundefined {\uparrow}

% - `cutrank`: cut rank

\DeclareDocumentCommand \cutrank { m }{\fn{cr}(#1)}

% - `maxrank`: max rank

\DeclareDocumentCommand \maxrank { m }{\fn{mr}(#1)}

% ### Particular sets

% - Natural numbers: `\Nat`
\DeclareDocumentMacro \Nat {\mathbb{N}}

% - Integers: `\Int`
\DeclareDocumentMacro \Int {\mathbb{Z}}

% - Positive integers: `\PosInt`
\DeclareDocumentMacro \PosInt {\mathbb{Z}^+}

% - Real numbers: `\Real`
\DeclareDocumentMacro \Real {\mathbb{R}}

% - Rational numbers: `\Rat`
\DeclareDocumentMacro \Rat {\mathbb{Q}}

% - The set $\{0, 1\}$: `\Bin`
\DeclareDocumentMacro \Bin {\mathbb{B}}

% - Identity relation: `\Id{X}`
\DeclareDocumentCommand \Id { m } {\mathord{\mathrm{Id}_{#1}}}