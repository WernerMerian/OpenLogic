% Other Enviroments
% -----------------

% OLP texts make use of a number of environments to encapsulate types
% of discursive text.  By default, these environments simply print
% their content without any special treatment. If you want to typeset
% any of them differently, you can change the definition of the
% environment here. E.g., you might want digressions in asmaller font
% and indented.  Refer to the \LaTeX\ documentation for how to
% accomplish this.

% OLP will provide two custom mechanisms for treating environments:
% supressing and deferring. Environments can be supressed by
% redefining them as the `comment` environment. For instance, to not
% print any digressions at all, add the lines
% ```
% \let\digress\comment
% \let\enddigress\endcomment
% ```
% after the definition of `digress`.

% - `explain`: Any explanatory material that's useful especially
%   on first reading but might be left out if the material is
%   presented only for review or completeness

\newenvironment{explain}{}{}

% - `informal`: Any informal explanation that's useful for
% novice readers but could be left out

\newenvironment{informal}{}{}

% - `intro`: for comments and comparisions to other introductory
% texts, e.g., regarding terminology, conventions, or proof methods.

\newenvironment{intro}{}{}

% - `pedantic`: for pedantic comments which users may want to exclude.

\newenvironment{pedantic}{}{}

% - `digress`: for digressions

\newenvironment{digress}{}{}

% - `history`: for historical notes and remarks

\newenvironment{history}{\paragraph{Historical Remarks}\par}{}

% - `reading`: for further reading

\newenvironment{reading}{\paragraph{Further Reading}\par}{}

% - `editorial`: for editorial remarks, warnings about missing parts, etc.

\newenvironment{editorial}{\begin{framed}}{\end{framed}}

\excludeenv{editorial} % exclude editorial remarks by default

% - `defish`: for definition-like material, e.g., rules

\newenvironment{defish}{\begin{oframed}\noindent}{\end{oframed}}

% - `derivation`: for derivations, a tabular environment with three
% colums for line number, formula, and justification.

\newenvironment{derivation}{%
	~\begin{trivlist}\item\begin{tabular}[b]{@{}rll@{}}}
		{\end{tabular} \end{trivlist}}
