% % The Default OLP Theorems File `open-logic-envs.sty`
% % OpenLogic Project
%
% Description
% ===========
%
% This file contains all theorem and other environments that are meant to
% be configured, changed, or adapted by a user generating their own
% text based on OLP text.  **Do not edit this file to customize your
% OLP-derived text!** A file `myversion.tex` adapted from
% `open-logic-complete.tex` (or from any of the contributed example
% master files) will include `myversion-thm.sty` if it exists, and
% otherwise will include this file. So if you'd like to make changes,
% instead **copy this file to `myversion-config.sty` and edit for
% customization**.

\NeedsTeXFormat{LaTeX2e}

% Theorem-like Environments
% -------------------------

% Theorem-like environments are provided for theorem (`thm`),
% definition (`defn`), lemma (`lem`), proposition (`prop`), corollary
% (`cor`), example (`ex`), problem (`prob`), remark (`rem`), and
% convention (`conv`). The definitions make use of the `thmtools`
% package.  To change the numbering or style of these environments,
% adjust their definitions. Refer to the [`thmtools`
% documentation](http://www.ctan.org/pkg/thmtools).

\declaretheorem[
	style        = plain,
	name         = Theorem,
	numberwithin = chapter,
]{thm}

\declaretheorem[
	style       = definition,
	name        = Example,
	sibling     = thm,
]{ex}

\declaretheorem[
	style       = plain,
	name        = Lemma,
	refname     = {Lemma, Lemmata},
	sibling     = thm,
]{lem}

\declaretheorem[
	style       = plain,
	name        = Proposition,
	sibling     = thm,
]{prop}

\declaretheorem[
	style       = plain,
	name        = Corollary,
	refname     = {Corollary, Corollaries},
	sibling     = thm,
]{cor}

\declaretheorem[
	style       = definition,
	name        = Definition,
	sibling     = thm,
]{defn}



\declaretheorem[
	style        = definition,
	name         = Axiom,
	unnumbered,
]{axiom}

\declaretheorem[
	style        = remark,
	name         = Remark,
]{rem}

\declaretheorem[
	style        = remark,
	name         = Note,
]{note}

\declaretheorem[
	style        = remark,
	name         = Case,
]{case}

\declaretheorem[
	style        = remark,
	name         = Convention,
]{conv}

% These are the names of the environments that `\cref` adds
% automatically. `thmtools` should do that as well, but sometimes it
% doesn't work, so we set them explicitly.  Redefine, e.g., if you
% want references to "Prop. 5.4"

\crefname{thm}{Theorem}{Theorems}
\crefname{ex}{Example}{Examples}
\crefname{defn}{Definition}{Definitions}
\crefname{lem}{Lemma}{Lemmata}
\crefname{prop}{Proposition}{Propositions}
\crefname{prob}{Problem}{Problems}
\crefname{rem}{Remark}{Remarks}
\crefname{figure}{Figure}{Figures}
\crefname{table}{Table}{Tables}

