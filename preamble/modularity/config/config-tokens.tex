% Tokens
% ======

% The following terms are *tokenized* throughout OLP. This means that
% by changing the definition in the configuration file, the term as
% printed will also change. This is simpler than searching and
% replacing these terms in all OLP texts, and it will also treat
% plurals as well as occurrences of the term at the beginning of a
% sentence (where it should be capitalized) correctly.

% Tokens are defined using the `\settexttoken` command.
% ```
% \settexttoken{token}{singular}{plural}[Singular][Plural]
% ```
% Here `token` is the term as it is used in the source text, where it
% typically is used as `!!{token}`.  `singular` and `plural` are the
% text you want printed wherever an OLP text contains `!!{token}` or
% `!!{token}s`.  In sentence-initial position, OLP texts would use
% `!!^{token}` and `!!^{token}s` to create capitalized versions of
% these token replacements.  By default, they are generated from
% `singluar` and `plural` by capitalizing the first character, but can
% be provided explicitly to `\settexttoken` as optional arguments.
% `!!a{token}` produces an indefinite article plus the token
% replacement. This is usually "a" unless the settexttoken command is
% used with a star, as in `\settexttoken{element}*{element}{elements}`,
% in which case it produces "an". This can be combined with `^` to
% produce the uppercase version, e.g., `!!^a{element}` for "An
% element".  `\article{token}` and `\Article{token}` produce just the
% article by itself (lower or uppercase, respectively).

% - `language`: defaults to "language", redefine for, e.g.,
% "signature"

\settexttoken{language}{language}{languages}

% - `formula`: defaults to "formula/formulas", redefine for
% plural "formulae," or for "wff".

\settexttoken{formula}{formula}{formulas}

% - `subformula`: defaults to "subformula", redefine for plural
% "subformulae," hyphenated spelling "sub-formula", or "sub-wff".

\settexttoken{subformula}{subformula}{subformulas}

% - `sentence`: defaults to "sentence", redefine for
% "closed formula" etc.

\settexttoken{sentence}{sentence}{sentences}

% - `variable`: defaults to "variable/variables", redefine to be more
% specific, e.g., "individual variable", "object veriable".

\settexttoken{variable}{variable}{variables}

% - `propositional variable`: defaults to "propositional
% variable/variables", redefine for, e.g., "sentence
% letter"

\settexttoken{propositional variable}{propositional variable}{propositional variables}

% - `constant`: defaults to "constant", redefine for "individual
% constant", "constant symbol."

\settexttoken{constant}{constant symbol}{constant symbols}

% - `predicate`: defaults to "predicate symbol".

\settexttoken{predicate}{predicate symbol}{predicate symbols}

% - `function`: defaults to "function symbol".

\settexttoken{function}{function symbol}{function symbols}

% - `operator`: defaults to "logical operator", redefine for "connective".

\settexttoken{operator}{logical operator}{logical operators}

% - `main operator`: defaults to "main operator", redefine for
% "outermost operator".

\settexttoken{main operator}{main operator}{main operators}

% - `free for`: defaults to "free for", redefine for
% "substitutable for" as in Enderton.

\settexttoken{free for}{free for}{free for}

% - `identity`: defaults to "identity predicate", redefine for
% "equality predicate."

\settexttoken{identity}*{identity predicate}{identity predicates}

% - `conditional`: defaults to "conditional", redefine for
% "implication."

\settexttoken{conditional}{conditional}{conditionals}

% - `biconditional`: defaults to "biconditional", redefine for
% "equivalence."

\settexttoken{biconditional}{biconditional}{biconditionals}

% - `falsity`: name of the falsity symbol, defaults to "falsity",
% redefine for "absurdity."

\settexttoken{falsity}{falsity}{falsities}

% - `truth`: name of the truth symbol, defaults to "truth", redefine
% for "verum" or "top".

\settexttoken{truth}{truth}{truth}

% - `structure`: term for first-order structures, defaults to
% "structure", redefine for "interpretation", "model".

\settexttoken{structure}{structure}{structures}

% - `valuation`: valuation, truth-value assignment

\settexttoken{valuation}{valuation}{valuations}

% - `domain`: domain of a structure

\settexttoken{domain}{domain}{domains}

% - `value`: value (denotation) of a term

\settexttoken{value}{value}{values}

% - `relational model`: term for modal structures, defaults to
% "relational model", redefine for "Kripke model".

\settexttoken{relational model}{relational model}{relational models}

% - `derivation`: derivation in a calculus, proof

\settexttoken{derivation}{derivation}{derivations}

% - `derive`: derive in a calculus, prove

\settexttoken{derive}{derive}{derives}

% - `derivable`: derivable in a calculus, provable

\settexttoken{derivable}{derivable}{derivable}

% - `derivability`: derivability in a calculus, provability

\settexttoken{derivability}{derivability}{derivabilities}

% - `nonderivability`: derivability in a calculus, unprovability

\settexttoken{nonderivability}{non-derivability}{non-derivabilities}

% - `tableau`: tableau, tableaux; or truth tree

\settexttoken{tableau}{tableau}{tableaux}

% - `signed formula`: signed formula in a tableau

\settexttoken{signed formula}{signed formula}{signed formulas}

% - `complete`: negation complete, syntactically complete (of theories)

\settexttoken{complete}{complete}{complete}

% - `axiomatizable`: effectively/recursively axiomatizable

\settexttoken{axiomatizable}*{axiomatizable}{axiomatizable}
\settexttoken{axiomatized}*{axiomatized}{axiomatized}
\settexttoken{axiomatizability}*{axiomatizability}{axiomatizability}

% - `represents`, `representable`

\settexttoken{represents}{represents}{represent}
\settexttoken{representable}{representable}{representable}

% - `discharge`: discharge an assumption in a natural deduction proof,
% cancel, close. Also: undischarged, non-cancelled, open.

\settexttoken{discharge}{discharge}{discharges}
\settexttoken{discharged}{discharged}{discharged}
\settexttoken{undischarged}*{undischarged}{undischarged}

% - `complete`: syntactic completeness of a set of sentences

\settexttoken{complete}{complete}{complete}

% - `enumerable`: term for finite or countably infinite; defaults to
% "enumerable", redefine for "countable".

\settexttoken{enumerable}*{enumerable}{enumerable}

% - `nonenumerable`: term for uncountable; defaults to
% "non-enumerable", redefine for "uncountable".

\settexttoken{nonenumerable}{non-enumerable}{non-enumerable}

% - `denumerable`: term for countably infinite; defaults to
% "denumerable".

\settexttoken{denumerable}{denumerable}{denumerable}

% - `element`: element of a set; redefine for "member"

\settexttoken{element}*{element}{elements}

% - `injective`, `injection`: redefine for "one-one" and "one-one
% function"

\settexttoken{injective}*{injective}{injective}
\settexttoken{injection}*{injection}{injections}

% - `surjective`, `surjection`: redefine for "onto" and "onto function"

\settexttoken{surjective}{surjective}{surjective}
\settexttoken{surjection}{surjection}{surjections}

% - `bijective`, `bijection`: redefine for "one-one onto" and "one-one
% onto function" or "correspondence"

\settexttoken{bijective}{bijective}{bijective}
\settexttoken{bijection}{bijection}{bijections}

% - `decidable`:

\settexttoken{decidable}{decidable}{decidable}

\settexttoken{computably enumerable}{computably enumerable}{computably enumerable}
\settexttoken{c.e.}{c.e.}{c.e.}

\settexttoken{argument}*{argument}{arguments}
\settexttoken{parameter}{parameter}{parameters}


% - `lambda define`: Also: represent.

\settexttoken{lambda define}{$\lambd$-define}{$\lambd$-defines}[$\lambd$-Define][$\lambd$-Defines]
\settexttoken{lambda defined}{$\lambd$-defined}{$\lambd$-defined}[$\lambd$-Defined]
\settexttoken{lambda definable}{$\lambd$-definable}{$\lambd$-definable}[$\lambd$-Definable]





% Colors
% ======

\settexttoken{colorC}{red}{red}
\settexttoken{colorD}{blue}{blue}
\settexttoken{colorE}{green}{green}
